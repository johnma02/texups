\documentclass[letterpaper,12pt]{article}
\setlength{\headheight}{15pt}
\setlength{\marginparwidth}{0pt}
\setlength{\marginparsep}{0pt} % width of space between body text and margin notes
\setlength{\evensidemargin}{0.125in} % Adds 1/8 in. to binding side of all 
% even-numbered pages when the "twoside" printing option is selected
\setlength{\oddsidemargin}{0.125in} % Adds 1/8 in. to the left of all pages when "oneside" printing is selected, and to the left of all odd-numbered pages when "twoside" printing is selected
\setlength{\textwidth}{6.375in} % assuming US letter paper (8.5 in. x 11 in.) and side margins as above
\raggedbottom
\setlength{\parskip}{\medskipamount}


\usepackage{amsmath, amsthm, amssymb, fancyhdr, enumitem, tikz, pgfplots, float}

\pgfplotsset{compat=1.18}

\pagestyle{fancy}
\lhead{LING101 --- HW3}
\begin{document}
\begin{enumerate}
    \item Each of the following words below describes a semantic relationship between words.
        However, the last part of each sentence is left blank. Complete them appropriately.
    \begin{enumerate}[label=(\alph*).]
        \item \textit{Irish setter, dalmatian, cocker spaniel are hyponyms of} \underline{dog} 
        \item \textit{tabby, tom, Persian, alley are hyponyms of} \underline{cat}
        \item \textit{dog, cat, goldfish, parakeet, hamster are hyponyms of} \underline{pet}
        \item \textit{knife, fork, spoon are hyponyms of} \underline{utensil}
        \item \textit{true is the antonym of} \underline{false}
        \item \textit{inaccurate is the antonym of} \underline{accurate}
        \item \textit{sister is the converse of} \underline{brother}
        \item \textit{teacher is the converse of} \underline{student}
        \item \textit{increase is the reverse of} \underline{decrease}
        \item \textit{deflate is the reverse of} \underline{inflate}
    \end{enumerate}
    \item Decide whether the following sentences are true or false.
    \begin{table}[htpb]
        \begin{tabular}{l r}
        \qquad(a). The reference of a sentence is the proposition expressed by it. & \underline{T}\\
        \qquad(b). The sense of a sentence is the proposition expressed by it. & \underline{F}\\
        \qquad(c). The reference of a sentence is some individual in the world. & \underline{F}\\
        \qquad(d). The reference of a sentence is not a set of individuals in the world. & \underline{T}\\
        \qquad(e). The reference of a sentence is a truth value. & \underline{F}\\
        \end{tabular}
    \end{table}
\item Observe the following humorous exchanges. Identify which maxim of conversation is violated in each exchange
    and explain how.
    \begin{enumerate}[label=(\alph*).]
        \item \textbf{Mother:} What are these cookie crumbs doing in your bed?


            \textbf{Child:} Nothing, they're just lying there.
        
            \paragraph{Answer:}
            The \textbf{maxim of relevance} is violated in this exchange, because the Child does
            not provide an answer which is connected to the conversation.


            
        \item \textbf{Owner:} If cats ruled the world, everyone would sleep on a pile of fresh laundry.


            \textbf{Cat:} Cats \textit{don't} rule the world?
            \paragraph{Answer:}
            The \textbf{maxim of relevance} is violated in this exchange, because the Owner makes a hypothetical statement,
            but the cat undermines the truthiness of the statement, rather than acknowledging the hypothetical.
        \item \textbf{Fire Department Operator:} Where's the phone that you are calling from?


            \textbf{Child:} On the wall.
            \paragraph{Answer:}
            The \textbf{maxim of quantity} is violated in this exchange, because the Child is not concise in their response to
            the Operator, or, they are not providing enough information about the location of their phone.
    \end{enumerate}
\item \underline{Underline} any deictic words in the following sentences.
    \begin{enumerate}[label=(\alph*).]
        \item I saw \underline{her} standing \underline{there}.
        \item The treasure chest is to \underline{your} \underline{right}.
        \item The name of \underline{this} rock band is "The Beatles". 
        \item The Declaration of Independence is signed \underline{last} year.
        \item \underline{These} are the times \underline{that} try men's soul.
        \item \underline{Yesterday}, all my troubles seemed so \underline{far} away.
        \item Sally always comes \underline{here}.
    \end{enumerate}

\item Match the following terms with their senses/references.
    \begin{table}[htpb]
        \begin{tabular}{l c l}
            (a). The sense of \textit{Adele} & \underline{f} & The truth value of a sentence\\
            (b). The reference of \textit{Adele} & \underline{c} & The mental representation of \textit{jellyfish}\\
            (c). The sense of \textit{jellyfish} & \underline{e} & The proposition expressed by the sentence\\
            (d). The reference of \textit{jellyfish} & \underline{b} & The individual singer named \textit{Adele}\\
            (e). The sense of \textit{Skating is dangerous} & \underline{d} & The set of all jellyfishes in the world\\
            (f). The reference of \textit{Skating is dangerous} & \underline{a} & The mental rep. associated with \textit{Adele}\\
        \end{tabular}
    \end{table}

\end{enumerate}
\end{document}



