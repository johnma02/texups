\documentclass[letterpaper,12pt]{article}
\setlength{\headheight}{15pt}
\setlength{\marginparwidth}{0pt}
\setlength{\marginparsep}{0pt} % width of space between body text and margin notes
\setlength{\evensidemargin}{0.125in} % Adds 1/8 in. to binding side of all 
% even-numbered pages when the "twoside" printing option is selected
\setlength{\oddsidemargin}{0.125in} % Adds 1/8 in. to the left of all pages when "oneside" printing is selected, and to the left of all odd-numbered pages when "twoside" printing is selected
\setlength{\textwidth}{6.375in} % assuming US letter paper (8.5 in. x 11 in.) and side margins as above
\raggedbottom
\setlength{\parskip}{\medskipamount}


\usepackage{amsmath, amsthm, amssymb, fancyhdr, enumitem, tikz, pgfplots}

\pgfplotsset{compat=1.18}

\pagestyle{fancy}
\lhead{MATH350 --- Exponential Distributions}
\begin{document}
\paragraph{Exponential Distributions:}

\paragraph{}
\textbf{Exponential Distributions} model the waiting time between arrivals.


\begin{itemize}
    \item The time you wait from one earthquake to the next.
    \item The time between customer arrivals.
\end{itemize}


The exponential distribution is the \textbf{continuous counterpart of the geometric distribution.}


\paragraph{Definition:}


Let $0 < \lambda < \infty$. A random variable $x$ has exponential distribution with parameter $\lambda$
if $x$ has the density function 

\[ f(x) = \begin{cases}
    \lambda e^{-\lambda x} & x \geq 0\\
    0 & x < 0
    \end{cases}
\]


$x \sim \mathrm{Exp}(x)$. 


$x$ is exponential with \textbf{rate} $\lambda$.


$x$ has the c.d.f. 

\begin{align*}
    F(t) &= P(x \leq t)\\
         &= \int_{-\infty}^t f(x)\, dx \\
         &= 0 + \int_0^t \lambda e^{-\lambda}\, dx\\
         &= \begin{cases}
             -e^{-\lambda x}\,\, \bigg|_0^t & t \geq 0\\
             0 & t < 0
             \end{cases}\\
         &= \begin{cases}
             1-e^{-\lambda t} & t \geq 0\\
             0 & t < 0
             \end{cases}
\end{align*}

\begin{align*}
    P(\frac{x}{2} > t) &= P(x > 2t) \\
                       &= \int_{2t}^{\infty} \lambda e ^{-\lambda x}\\
                       &= e^{-2\lambda t}
\end{align*}

\paragraph{Example:} The length of a call, $T$  is modeled by an exponential with a mean length of
10 minutes. What is $P(T > 8)$? What is $P(8 \le T \le 22)$.

\paragraph{Solution:}


We have $E(T) = 10$, then $\lambda = \frac{1}{10}$.


\begin{align*}
    P(T > 8) &= \frac{1}{10} \int_8^{\infty} e^{\frac{x}{10}}dx \\
             &= e^{-\frac{1}{10}\cdot 8}\\
             &\approx 0.4493
\end{align*}

\begin{align*}
    P(8 \le T \le 22) &= P(T \ge 8) - P(T > 22)\\
                      &= e^{-\frac{8}{10}} - e^{-\frac{22}{10}}\\
                      &\approx 0.3385
\end{align*}

\paragraph{Example:} There is a certain protein in human cells which takes $T$ time to be destroyed.
The common assumption in biochemistry is that $T \sim \exp (\lambda n)$, where $\lambda > 0$.


Suppose $n$ is the number of copies of this protein in a given cell. What is the value of $n$ such that
$P(T > 10^{-2}) < \frac{1}{10}$.

\paragraph{Example:} An alarm clock is rigged to ring at some time $T$, such that $T \sim \exp(\frac{1}{3})$.
You've waited for 7 hours, and it hasn't been set off yet. What's the probability that you will wait
$x$ more hours.

\paragraph{Solution:} $P(T \le 7+x \,|\, T \ge 7)$.
By memorylessness,
\begin{align*}
    P(T \le 7+x \,|\, T \ge 7) &= P(T > x)\\
                               &= e^{-\frac{x}{3}}\\
\end{align*}

\paragraph{Example:} Suppose $x \sim \exp(2)$. Find the real number $a < 1$ such that
the events $\{x \in [0,1]\}$, and $\{x \in [a,2]\}$ are independent.
\paragraph{Solution:} We want
\[
    P(x \in [0,1] \cap x \in [a,2]) = P(x\in[0,1]) \cdot P(x \in [a,2]) 
\]


We know $P(x \in [0,1]) = 1-e^{-2}$.


\[
    P(x \in [a,2]) = \begin{cases}
        e^{-2} - e^{-4} & a < 0 \\
        e^{-2a} - e^{-4} & 0 \le a < 1\\
        \end{cases}
\]

\[
    P(x \in [0,1] \cap x \in [a,2]) = \begin{cases}
        1-e^{-2} & a < 0\\
        e^{-2a} - e^{-2} & 0 \le a < 1\\
    \end{cases}
\]
\begin{align*}
    (1-e^{-2})(e^{-2a}-e^{-4}) &= e^{-2a} - e^{-2}\\
    e^{-2a} - e^{-2-2a}-e^{-4}+e^{-6} &= e^{-2a} - e^{-2}\\
    e^{-2-2a} &= e^{-2}-e^{-4}+e^{-6}\\
    e^{-2a} &= 1-e^{-2}+e^{-4}\\
\end{align*}
\end{document}



