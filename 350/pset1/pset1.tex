\documentclass{article}

\usepackage{amsmath,amsthm,amssymb,fancyhdr}
\pagestyle{fancy}
\lhead{MATH350: HW1}

\begin{document}

\emph{A ”standard” deck of playing cards consists of 52 Cards in each of the 4 suits of Spades,
Hearts, Diamonds, and Clubs. Each suit contains 13 cards: Ace, 2, 3, 4, 5, 6, 7, 8, 9, 10,
Jack, Queen, King. A ”hand” is a selection of five cards - thus there are $\binom{52}{5}$ possible hands.}

\vspace{5mm}

1. A “flush” in poker is a hand of 5 cards of the same suit. Calculate the number of possible
flushes. A “full house” is a hand consisting of three cards with the same number/face
value and two cards of the same number/face value. Calculate the number of possible
full house hands.
\vspace{5mm}

Suppose we chose a suit. There are 4 of these to choose from.
Then we have 13 cards of the same suit to choose from. We choose five from these, giving us
a total number of 
\begin{align*}
    4 \cdot \binom{13}{5}
\end{align*}
different flushes we can possible draw.

\vspace{5mm}

For a full house, we may fix some denomination to pick three cards from, and we may fix a different
denomination to pick two cards from. For our first denomination, we choose 3 cards from 4, and for our second 
denomination, we choose 2 cards from 4. We have a total of
\begin{align*}
13 \cdot 12 \cdot \binom{4}{3} \binom{4}{2}
\end{align*}

different full houses we may draw.

\vspace{5mm}

2. Consider $n$ digit numbers using the digits 0, 1, . . . , 9. How many such numbers are there
that have at least one pair of consecutive digits that are equal?

\vspace{5mm}

Rather than directly counting the number of $n$ digit numbers that satisfy this condition, we should
consider the number of $n$ digit numbers which do not satisfy this condition. In other words,
we want to find the number of $n$ digit numbers which do not have any repeated pairs.

\vspace{5mm}

We fix an initial digit for $n_0$. We have 10 digits to choose from (we allow leading zeros). 
For each successive digit, we have 9 possible digits to choose from, considering that we may not choose
the previous digit.

\vspace{5mm}

Thus, we have a total of 
\begin{align*}
10 \cdot 9^{n-1}
\end{align*}
$n$ digit numbers which satisfy this condition.

\vspace{5mm}

3. Consider an $m$ x $n$ size checkerboard ($m$ rows and $n$ columns). How many paths are
there using only steps going ’up’ or ’right’ from (0, 0) to $(n, m)$? Suppose $m$ and $n$ are
even. How many paths go through the point $(n/2, m/2)$?
\vspace{5mm}

For an $m$ x $n$ checkerboard, to travel from the bottom left position to the top right position
using only upward steps of 1 and rightward steps of 1 deterministically takes $n+m$ steps. We must
travel exactly $n$ units right and $m$ units up. 

\vspace{5mm}

Let $t = n+m$, then suppose then we have some sequence of moves $a_1, \ldots, a_t$ 
representing the moves we take to go from (0,0) to ($n, m$). $a_i$ has value 1 if we have moved
upwards on $a_i$ or 0 if we have moved right on $a_i$.

\vspace{5mm}

We know exactly $n$ of these $a_i$ are equal to
one. So the calculation of the number of unique paths is equivalent to choosing $n$ of these
$a_i$ to be equal to 0 out of $t$.

\vspace{5mm}

Thus, the total number of unique paths we have is equal to

\begin{align*}
    \binom{n+m}{n}
\end{align*}

By our binomial properties, we also know that

\begin{align*}
    \binom{n+m}{n} &= \binom{n+m}{n+m - n}\\
                       &= \binom{n+m}{m}
\end{align*}

\vspace{5mm}

Supposing $n$, $m$ are even, the computation of paths from (0,0) to $(\frac{n}{2}, \frac{m}{2})$ is very similar.
We must travel exactly $\frac{n}{2} + \frac{m}{2}$ steps to arrive to our destination
$(\frac{n}{2}, \frac{m}{2})$ We again reduce our problem to choosing $\frac{n}{2}$ of 
$\frac{n}{2} + \frac{m}{2}$ steps to be rightward. Then, from $(\frac{n}{2}, \frac{m}{2})$, we must
travel to $(n, m)$. We again must travel $\frac{m}{2}$ steps up and $\frac{n}{2}$ steps right.


\begin{align*}
    \bigg(\binom{\frac{n}{2}+\frac{m}{2}}{\frac{n}{2}}\bigg)^{2}
\end{align*}

4. In a state lottery, 7 numbers are picked from 1, \ldots, 40 uniformly. Describe a sample
space $\Omega$ and probability measure $\mathbb{P}$ model this experiment. What is the probability
that out of the seven chosen numbers, exactly 4 are odd?

\vspace{5mm}

\emph{With replacement, assuming order matters}

\vspace{5mm}

Our sample space is the set of all potential ordered 7-tuples of numbers $\in$ [1, 40], with replacement. There are $40^7$ 7-tuples
in our sample space. Assuming uniformity, the probability measure of any one lottery result is 
$\frac{1}{40^{7}}$

\vspace{5mm}

In general, we may choose four numbers in this lottery drawing to be odd, and choose only even numbers for the
remaining three drawings. In [1, 40], there are 20 even numbers and 20 odd numbers. We have then the
probability of any drawing containing exactly four odd numbers equivalent to
\begin{align*}
\frac{\binom{7}{4}\cdot20^{3}\cdot20^{4}}{40^{7}} &= \frac{\binom{7}{4}\cdot20^{7}}{40^{7}}
\end{align*}
\vspace{5mm}

\emph{Without replacement, asuming order matters}

\vspace{5mm}

Our sample space is the set of all potential ordered 7-tuples of numbers $\in$ [1, 40] without replacement. There are
$P(40, 7) = \frac{40!}{33!}$ total 7-tuples in our sample space, and the probability of any one lottery
drawing appearing is $\frac{1}{P(40, 7)}$. We can follow the same process of choosing 4 drawings to be odd,
and from there, counting the number of permutations of odd integers we can form. We have then that the probability of any
drawing containing exactly four odd numbers equivalent to 
\begin{align*}
    \frac{\binom{7}{4}\cdot\frac{20!}{16!}\cdot\frac{20!}{17!}}{P(40, 7)}
\end{align*}

\vspace{5mm}

\emph{Without replacement, assuming order does not matter}

\vspace{5mm}

Our sample space is the set of all potential 7-tuples of numbers $\in [1,40]$ without replacement.
The size of our sample space is $\binom{40}{7}$. We may choose 4 of our drawings to be odd, giving us

\begin{align*}
    \mathbb{P}(\text{Exactly four odd numbers are chosen})&=\frac{\binom{20}{3}\cdot\binom{20}{4}}{\binom{40}{7}}
\end{align*}
\vspace{5mm}

5. We repeatedly roll a fair die until the number six appears and then we stop. The outcome
of the experiment is the number of rolls. Describe a sample space $\Omega$ and probability
measure $\mathbb{P}$ to model this experiment. Calculate the probability that the number six
never appears.

\vspace{5mm}

Our experiment has a non-discrete probability measure, and $|\Omega|$ is uncountable. Let's define $\Omega$
to be equal to $\{1, 2, 3, \ldots , \infty\}$, where $\omega \in \Omega$ represents the event of not rolling
6 on rolls $1$ to $\omega$, and rolling 6 on roll $\omega$.

\vspace{5mm}

Assuming uniformity of our rolls, the probability of not rolling 6 on any given roll is $\frac{5}{6}$,
and the probability of rolling 6 on roll $\omega$ is $\frac{1}{6}$. It
follows that the probability of any $\omega$ is equal to 
\begin{equation}
    \mathbb{P}(\omega) = \bigg(\frac{5}{6}\bigg)^{\omega} \cdot \frac{1}{6}
\end{equation}
To find the probability that 6 never appears, we must find $\mathbb{P}(\infty)$. We can derive 
$\mathbb{P}(\infty)$ using the axioms of probability.
\begin{align*}
    1 &= \mathbb{P}(\Omega)\\
      &= \mathbb{P}(\infty) + \sum_{\omega=1}^{\infty} \mathbb{P}(\omega)
\end{align*}
using (1),
\begin{align*}
    \sum_{\omega=1}^{\infty} \mathbb{P}(\omega) &= \sum_{\omega=1}^{\infty} \bigg(\frac{5}{6}\bigg)^{\omega} \cdot\frac{1}{6}\\
                                                &= \frac{1}{6}\bigg[\bigg(\frac{5}{6}\bigg)^{1} + \bigg(\frac{5}{6}\bigg)^{2} + \ldots + \bigg(\frac{5}{6}\bigg)^{\infty}\bigg]\\
                                                &= \frac{1}{6}\cdot\frac{1}{1-\frac{5}{6}}\\
                                                &= 1
\end{align*} 
Thus, the probability of never rolling a 6 is equal to 0.

\vspace{5mm}

6. A drawer contains 12 pairs of socks. In the experiment, 8 are chosen at random. Describe
a sample space $\Omega$ and probability measure $\mathbb{P}$ to model this experiment. What is the
chance that there will be exactly one complete pair?

\vspace{5mm}

Our sample space is the set of all 8 sock drawings from 24. The probability of any one sock drawing appearing
is $\frac{1}{24C8}$. 

\vspace{5mm}

Suppose we fix a sock pair to be drawn. We have 12 such pairs to choose from. When
we are done fixing this pair, we have 22 socks left to choose from, but for each sock we choose, we must
discard its complement sock pair. We have then, 11 sock pairs to choose from. For each sock pair, we have
two socks to choose from. From these 11 sock pairs, we choose 6 socks. 
Therefore, the probability measure of drawing one complete sock pair is

\begin{align*} 
    \frac{12\cdot2^{6}\cdot\binom{11}{6}}{\binom{24}{8}}
\end{align*}

\vspace{5mm}

7. Suppose $P(A) = 0.4$ and $P(B) = 0.7$. With no further assumptions on $A$ and $B$, show
that $0.1 \leq P(A \cap B) \leq 0.4$.

\vspace{5mm}

By P.I.E

\begin{align*}
    P(A \cup B) = P(A) + P(B) - P(A \cap B)\\
\end{align*}

By axioms of probability

\begin{align*}
    1 \geq P(A \cup B)
\end{align*}

implying

\begin{align*}
    1 &\geq 0.4 + 0.7 - P(A \cap B)\\
      &\geq 1.1 - P(A \cap B) \\
    1-1.1 &\geq -P(A \cap B) \\ 
    0.1 &\leq P(A \cap B)
\end{align*}

finally, because $P(A \cap B) \subseteq P(A)$, by the monotonicity of probability, 

\begin{align*}
    0.1 \leq P(A \cap B) \leq 0.4
\end{align*}

\vspace{5mm}

8. Two cards are dealt from a shuffled standard deck of cards. This means the cards are
sampled uniformly at random without replacement. What is the probability that both
cards are aces and one of them is the ace of spades?

\vspace{5mm}

There are only three hands which satisfy these conditions. There are $\binom{52}{2}$ possible hands.
Therefore the probability of drawing exactly the ace of spaces and one other ace is

\begin{align*}
    \frac{3}{\binom{52}{2}} = 3 \cdot \frac{2}{52\cdot51}
\end{align*}
\end{document}

