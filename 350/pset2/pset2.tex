\documentclass{article}

\usepackage{amsmath, amsthm, amssymb, fancyhdr}
\pagestyle{fancy}
\lhead{MATH350 - HW2}
\begin{document}
1. A box contains 3 marbles - 1 red, 1 green and 1 blue. In Experiment I, a marble is
drawn from the box, replaced and then a second marble is drawn. In Experiment II the
second marble is drawn without replacing the first marble. Describe the sample spaces
for Experiment I and Experiment II.

\paragraph{Experiment I:} Our sample space, $\Omega$, consists of all possible pairs of red, green,
and blue marbles. There are $3^{2}$ elements in $\Omega$. For any $\omega \in \Omega$, $\mathbb{P}(\omega)
= \frac{1}{3^2}$.

\paragraph{Experiment II:} Our sample space, $\Omega$, consists of all possible ways of choosing 2 colored
marbles of three distinct red, green, and blue marbles. There are $3\cdot2$ elements in $\Omega$. For
any $\omega \in \Omega$, $\mathbb{P}(\omega) = \frac{1}{3\cdot2}$.

\vspace{5mm}

2. An urn contains 4 balls numbered 1,2,3 and 4. I draw 4 balls with replacement and
note the numbers. What is the probability that at least one number appears exactly
two times?

\paragraph{} Our sample space, $\Omega$, consists of ordered tuples of integers $\in [1, 4]$. There are
$4^{4}$ elements in $\Omega$. We may decompose our event into disjoint events as follows. Firstly, we 
want to find the complement of our event, when no number appears exactly twice. This event may be
decomposed into the disjoint events that follow.
\begin{enumerate}
    \item The event that all numbers appear exactly once.
    \item The event that one number appears three times.
    \item The event that one number appears four times.
    \end{enumerate}

\paragraph{Event 1:} This is simply the number of permutations of 1, 2, 3, 4. We can enumerate this as 4!.
\paragraph{Event 2:} We can choose one number to fix three drawings, and choose another number for the fourth drawing.
We also need to choose the position of these three numbers to be fixed. 
In total, we can enumerate this as $4\cdot3 \cdot \binom{4}{3}$.
\paragraph{Event 3:} Trivially, there are four ways to choose this.

\vspace{5mm}

Thus, the probability of no number appearing exactly two times is

\begin{align*}
    \mathbb{P}(\omega) = \frac{4! + 4\cdot3\cdot\binom{4}{3} + 4}{4^{4}}
\end{align*}


We take 1 - $\mathbb{P}(w)$ to find our final answer. Using MATLAB or Octave, 
we calculate this probability to be equal to around .7031.

\vspace{5mm}

3. From 10 married couples how many ways are there to select a group of 6 people not
containing any of the couples?

\paragraph{}Assuming each relationship is monogamous, for every person we choose, we must remove their partner from contention. We can thus reduce our
enumeration process to choosing 6 people from 10. Additionally, we must consider each partner in a couple
distinct, i.e., for each pair we choose, we choose one of two partners. 

\paragraph{}Our final enumeration is equal to
\begin{align*}
    2^{6}\cdot\binom{10}{6} = 13440
\end{align*}

\vspace{5mm}

4. A 5-card hand is dealt from a well-shuffled deck of cards. What is the probability the
hand contains at least one card from each of the 4 suits? 

\paragraph{}\emph{There are 13 cards in a suit.}

\paragraph{}From each suit we choose a card, and we choose a suit to pick 2 cards from. 

\begin{align*}
    \mathbb{P}(\omega) = \frac{4\cdot\binom{13}{2}\cdot13^{3}}{\binom{52}{5}}
\end{align*}
\paragraph{}Using Octave, we find this probability equal to around .2637.

\vspace{5mm}

5. Pick a uniformly chosen point in a unit square and draw a circle of radius 1/3 around
the point. Find the probability that the circle lies entirely inside the square.


\paragraph{}\emph{A unit square has area 1.}
\paragraph{}Geometrically, the area which our point may form a completely encapsulated circle forms a square.
Specifically, this point may not be within within 1/3 of any of the unit square's edges. We take the length
of each edge, and subtract 2/3 from them to form our new square. The area of this square is 1/9. Thus, the
probability of this point forming a circle entirely inside the square is 
\begin{align*}
    \mathbb{P}(\omega) = 1/9
\end{align*}

\vspace{5mm}

6. You start with 4 dollar bills in a game. You roll a fair die repeatedly. Every time you
fail to get a 6, you lose one of your dollar bills. When you get your first 6, you get to
take the money that remains on the table. If the money runs out before you get a 6,
you lose the game and leave with no money. Let X be the amount you leave with. Find
the possible values and pmf of X.

\paragraph{}There are five outcomes to this game. We may leave the game with 0, 1, 2, 3, or 4 dollars.
There is only one way to arrive to each outcome by playing the game. Assuming we exit the game upon
rolling a six.
\begin{itemize}
    \item 0: We roll exactly 0 sixes from 4 rolls. This outcome has a $(\frac{5}{6})^{4} = .4823$ 
        chance to occur.
\item 1: We roll exactly 1 six from 4 rolls. This outcome has a $(\frac{5}{6})^{3}\cdot\frac{1}{6} = .0964$ chance
    to occur.
    \item 2: We roll exactly 1 six from 3 rolls. This outcome has a $(\frac{5}{6})^{2}\cdot\frac{1}{6} = .1157$ chance
    to occur.

    \item 3: We roll exactly 1 six from 2 rolls. This outcome has a $(\frac{5}{6})\cdot\frac{1}{6} = .1389$ chance
    to occur.

    \item 4: We roll exactly 1 six from 1 rolls. This outcome has a $\frac{1}{6} = .1667$ chance
    to occur.
 
    \end{itemize}
\paragraph{}Let $f(x)$ be the pmf of our experiment.

    \[ f(x) = \begin{cases} 
          .4823 & x = 0 \\
          .0964 & x = 1 \\
          .1157 & x = 2 \\
          .1389 & x = 3 \\
          .1667 & x = 4
       \end{cases}
    \]

\vspace{5mm}

7. There is a bin with 9 blue marbles and 4 yellow marbles. Draw 3 marbles without
replacement and record their colors. Put the marbles back in the bin. Perform this
procedure 20 times. Let X be the number of times that the 3 draws resulted in exactly
3 blue marbles. Find the pmf of X, \emph{identify its distribution by name and give its
parameters (Not required).}


\paragraph{}X may take values from 0 to 20. For any one time we draw three marbles, we have a $\frac{9}{13}
\cdot \frac{8}{12} \cdot \frac{7}{11} = \frac{42}{143}$ chance of drawing three blue marbles. There are $2^{20}$ possible
outcomes for the total experiment. 

\paragraph{}For 20 draws, suppose $k$ of these drawings succesfully result in drawing three marbles. For $k$ 
successful drawings, there are $\binom{20}{k}$ ways our experiment
can occur. Given that we know that $k$ of these experiments must have been successful, we must also have
$20-k$ unsuccessful experiments.

\paragraph{}Thus, for any given value for $k, 0 \le k \le 20$ we have a 
\begin{align*}
    \frac{(42)^{k}\cdot (101)^{20-k}}{(143)^{20}}\cdot \binom{20}{k}
\end{align*}
\paragraph{}chance for $k$ to occur.

\vspace{5mm}

8. Two fair dice are rolled. Let $M$ be the maximum of the two numbers and $m$ the minimum
of the two numbers on the dice.

\paragraph{(a).} Find the possible values of $M$ and $m$.

\paragraph{(b).} For every integer $k$, determine $\mathbb{P}(M \le k)$. Find the probability mass function of
$M$.
\paragraph{(c).} Find the pmf of $m$.

\paragraph{Solution: (a).} The possible values of $M$ and $m$ are 1, 2, 3, 4, 5, and 6.
\paragraph{Solution: (b).} We should establish the sample space of $M$ before we find the pmf of $M$.
For any value $\mu \in M$, we may find the corresponding rolls as follows.
\begin{enumerate}
    \item Choose $\mu_1$ to be the max of $\mu_1, \mu_2$
    \item Count the number of ways to choose $\mu_2$ such that $\mu_2 \le \mu_1$.
    \end{enumerate}
\paragraph{}Therefore, there are $2\mu - 1$ possible rolls which correspond to $\mu$. For example, $\mu = 3$.
We may roll (1,3), (2,3), (3,3), (3,2), and (3,1) to get $\mu = 3$.

\paragraph{}For 2 dice rolls, there are 36 possible outcomes.

\paragraph{}For $k \in \mathbb{Z}$, $1\le k\le6$, intuitively there are
\begin{align*}
    \sum_{i=1}^{k} 2i - 1
\end{align*}
\paragraph{}pairs which satisfy $M \le k$. By induction, we can prove that
\begin{align*}
    \sum_{i=1}^{k} 2i - 1 = k^{2}
\end{align*}
\begin{proof}
    \begin{align*}
        \displaystyle\sum_{i=1}^n 2i-1 = n^{2}
    \end{align*}
    \paragraph{}Let $n = 1$. 
    \begin{align*}
        2\cdot1 - 1 &= 1^{2}\\
        1&=1
    \end{align*}
    \paragraph{}Assume 
    \begin{align*}
        \sum_{i=1}^{k} 2i - 1 = k^{2}
    \end{align*}
    \paragraph{}then
     \begin{align*}
         \sum_{i=1}^{k+1} 2i - 1 &= (k+1)^{2}\\
         2(k+1)-1 + k^{2} &= k^2 +2k + 1\\
         2k + 2 - 1 + k^2 &= k^2 + 2k + 1\\
         k^2 + 2k + 1 &= k^2 + 2k + 1
    \end{align*}
\end{proof}
\paragraph{}Finally, let $f(k)$, for $k \in \mathbb{Z}$ denote the pmf of $M$, then

    \[ f(k) = \begin{cases} 
          0 & k < 1 \\
          \frac{k^2}{36} & 1 \le k \le 6\\
          1 & k > 6
       \end{cases}
    \]

\paragraph{Solution: (c).} Similarly to (b)., we should establish our sample space for $m$. We notice
that for $\mu \in m$, if we fix any such value $\mu$ to be the minimum of $\mu_x$ and $\mu_y$, 
there are $2(6-\mu)+1$ pairs which satisfy $\mu = \text{min}(\mu_x, \mu_y)$. For example, $\mu = 4$ corresponds to 
(4,4), (4,5), (5,4), (4,6), (6,4).

\paragraph{}Therefore, for $1 \le k \le 6$, $\mathbb{P}(k) = \frac{2(6-k)+1}{36}$
\end{document}


