\documentclass[letterpaper,12pt]{article}
\setlength{\headheight}{15pt}
\setlength{\marginparwidth}{0pt}
\setlength{\marginparsep}{0pt} % width of space between body text and margin notes
\setlength{\evensidemargin}{0.125in} % Adds 1/8 in. to binding side of all 
% even-numbered pages when the "twoside" printing option is selected
\setlength{\oddsidemargin}{0.125in} % Adds 1/8 in. to the left of all pages when "oneside" printing is selected, and to the left of all odd-numbered pages when "twoside" printing is selected
\setlength{\textwidth}{6.375in} % assuming US letter paper (8.5 in. x 11 in.) and side margins as above
\raggedbottom
\setlength{\parskip}{\medskipamount}

\usepackage{amsmath, amsthm, amssymb, fancyhdr, enumitem, tikz}
\usepackage{pgfplots}
\pagestyle{fancy}
\lhead{MATH350 --- Normal Distribution}
\begin{document}
   \paragraph{Gaussian / Normal Distribution} 
   \paragraph{}A random variable $Z$ has a standard distribution (std. Gaussian) if $Z$ has density
   \[
       \phi(x) = \frac{1}{\sqrt{2\pi}}e^{-\frac{x^2}{2}}
   \]

\newcommand\gauss[2]{1/(#2*sqrt(2*pi))*exp(-((x-#1)^2)/(2*#2^2))} % Gauss function, parameters mu and sigma

\begin{tikzpicture}
\begin{axis}[every axis plot post/.append style={
  mark=none,domain=-2:3,samples=50,smooth}, % All plots: from -2:2, 50 samples, smooth, no marks
axis x line*=bottom, % no box around the plot, only x and y axis
axis y line*=left, % the * suppresses the arrow tips
enlargelimits=upper] % extend the axes a bit to the right and top
\addplot {\gauss{0}{0.5}};
\addplot {\gauss{1}{0.75}};
\end{axis}
\end{tikzpicture}
\paragraph{Binomial:} $S_n \sim \mathrm{Bin}(n,p)$, where $S_n$ is the number of heads in $n$ independent
tosses of a coin, where $P(H) = p$.

\begin{align*}
    Y &= \frac{S_n - E(S_n)}{\sqrt{\mathrm{Var}(S_n)}}\\
      &= \frac{S_n - np}{\sqrt{np(1-p)}}
\end{align*}
\[
    S_n = X_1 + \ldots + x_n
\]
\paragraph{}where $x_i$ are Bernoulli random variables.
\paragraph{}\textbf{Sums of a large number of independent random variables are distributed with a normal
distribution.}
\paragraph{}We write 
\[
    Z \sim N(0,1)
\]
\paragraph{}Where $N$ denotes the normal distribution, the first parameter of our distribution is the mean
of the distribution, and the second parameter is the standard deviation.
\paragraph{cdf of $Z \sim N(0,1)$}
    \begin{align*}
        \Phi(x) = P(Z \le x) &= \int_{-\infty}^{x}\frac{1}{\sqrt{2\pi}}e^{-\frac{t^2}{2}}\,dt
    \end{align*}
\paragraph{}We cannot solve this integral in solved form.    
\paragraph{}The cdf is evaluated numerically for different values of $x$ to produce a normal table.

\paragraph{Ex:}$Z\sim N(0,1)$. Find $P(-1\le x\le 1.5)$.
\begin{align*}
    \int_{-1}^{1.5} \phi(x)\,dx&= \int_{-\infty}^{1.5}\phi(x)\,dx - \int_{\infty}^{-1}\phi(x)\,dx\\
                               &= \Phi(1.5) - \Phi(-1)\\
                               &= 0.9322-1+0.8413\\
                               &= 0.7745
\end{align*}
\paragraph{Ex:}Find the $c$ values such that $Z$ has approximately $\frac{2}{3}$ chance to be in [$-c,c$].
\begin{align*}
    \frac{2}{3} &= \Phi(c)- \Phi(-c)\\
                &= \Phi(c) - (1-\Phi(c))\\
                &= 2\Phi(c)-1\\
                &\to \Phi(c) = \frac{5}{6} = 0.833
\end{align*}
\paragraph{}The value closest to this is $c = 0.97$.

\paragraph{Defining Parameters for the Normal Distribution}
\paragraph{}Let $\mu$ be real and $\sigma > 0$. A random variable $x$ has the normal distribution
with mean $\mu$ and standard deviation $\sigma$ if $x$ has density
\[
    f(x) = \frac{1}{\sqrt{2\pi\sigma^2}}e^{\frac{-(x-m)^2}{2\sigma^2}}
\]
\paragraph{}We write $x \sim N(\mu, \sigma^2)$.
\begin{enumerate}
    \item $S_n \sim N(np, np(1-p))$ as $n \to \infty$.
\end{enumerate}
\paragraph{Quiz: 1, 2, 3, 4, 6, 7}
\end{document}



