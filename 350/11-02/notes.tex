\documentclass[letterpaper,12pt]{article}
\setlength{\headheight}{15pt}
\setlength{\marginparwidth}{0pt}
\setlength{\marginparsep}{0pt} % width of space between body text and margin notes
\setlength{\evensidemargin}{0.125in} % Adds 1/8 in. to binding side of all 
% even-numbered pages when the "twoside" printing option is selected
\setlength{\oddsidemargin}{0.125in} % Adds 1/8 in. to the left of all pages when "oneside" printing is selected, and to the left of all odd-numbered pages when "twoside" printing is selected
\setlength{\textwidth}{6.375in} % assuming US letter paper (8.5 in. x 11 in.) and side margins as above
\raggedbottom
\setlength{\parskip}{\medskipamount}

\usepackage{amsmath, amsthm, amssymb, fancyhdr, enumitem, tikz}
\pagestyle{fancy}
\lhead{MATH350 --- Lecture 11/02/2022}
\begin{document}
\paragraph{Expectation of Random Variables}
\paragraph{}Let $g$ be a real valued function defined on the range of a random variable $x$.
\paragraph{}If $x$ is discrete such that it has some probability mass function, $f$,

\[
    E(g(x)) = \sum_k g(k)\cdot P(x=k)
\]
\paragraph{}If $x$ is continuous with density $f$, 

\[
    E(g(x)) = \int_{-\infty}^{\infty}g(x)f(x)\,dx
\]

\paragraph{Example:}A stick of length $l$ is broken at a uniformly chosen point. What is the
expected length of the longer piece?

\paragraph{Sol:}We define a variable $L$ to be the length of the longer piece when we break the stick at
a point $x$. $x \sim U[0, l]$.

\[ L = \begin{cases}
    l-x & x < \frac{l}{2}\\
    x & x \ge \frac{l}{2}\\
    \end{cases}
\]
\begin{align*}
    E(L(x)) &= \int_{-\infty}^{\infty} L(x)f(x)\,dx\\
            &= \int_{0}^{\frac{l}{2}} L(x)f(x)\,dx + \int_{\frac{l}{2}}^{l}L(x)f(x)\,dx\\
            &= \int_{0}^{\frac{l}{2}} (l-x)\frac{1}{l}\,dx + \int_{\frac{l}{2}}^{l}x\frac{1}{l}\,dx\\
            &= \frac{(l-x)^2}{2l} \, \bigg|_{\frac{l}{2}}^0 + \frac{x^2}{2l}\, \bigg|_{\frac{l}{2}}^l\\
            &= \frac{3l}{4}
\end{align*}
\paragraph{Ex:}$C$ is the amount of money paid for an accident. The cost of the accident $Y \sim U[100,1500]$.
\begin{align*}
    &C = g(Y)\\
    &g(Y) = min(Y, 500)\\
    &g(Y) = \begin{cases}
        y & y < 500\\
        500 & y \ge 500 \\
        \end{cases}
\end{align*}
\paragraph{}What is $E(C) = E(g(Y))$?
\paragraph{Sol:}
\begin{align*}
    E(g(Y)) &= \int_{-\infty}^{\infty}g(Y)f(Y)\,dY\\
            &= \frac{1}{1400} \int_{100}^{1500} g(Y)\,dY\\
            &= \frac{1}{1400} \bigg( \int_{100}^{500} Y\,dY + \int_{500}^{1500}500\,dY\bigg)\\
            &= \frac{1}{1400} \bigg( \frac{250000-10000}{2} + 500\cdot 1000\bigg)\\
            &= \$442.86
\end{align*}
\paragraph{Variance, Standard Deviation}
\[
    SD = \sqrt{\frac{\displaystyle\sum_{i=1}^n (s_i - \hat{s})^2}{n}}
\]
\paragraph{}Let $x$ be a random variable with mean $E(x) = \mu$. The variance of $x$ is given by

\[
    E(x-\mu)^2
\]
\paragraph{}--- also denoted as $\sigma^2$
\paragraph{Discrete Variables}
\[
    E((x-\mu)^2) = \sum_k (k-\mu)^2 \cdot P(x=k)
\]
\paragraph{Ex:}$x \sim \mathrm{Ber}(p)$. What is Var($x$)?
\paragraph{Sol:}
\begin{align*}
    \mathrm{Var}(x) &= \sum_{k=0,1} (k-p)^2 \cdot P(x=k)\\
                    &= (0-p)^2(1-p) + (1-p)^2(p)\\
                    &= p(1-p)(p+1-p)\\
                    &= p(1-p)
\end{align*}
\paragraph{Note}
\begin{align*}
    \mathrm{Var}(x) &= E[(x-E(x))^2]\\
                    &= E(x^2) - (E(x))^2
\end{align*}

\paragraph{Linear Functions, Properties of Variance}
\paragraph{}Let $g$ be a linear function $g(x)= ax+b$.
\begin{align*}
    E[g(x)] &= E[ax+b]\\
            &= a\cdot E(x)+b\\
            &= E[ax] + E[b]\\
\end{align*}
\paragraph{}We also see that $E[cx] = c\cdot E[x]$, $c \in \mathbb{R}$, and
\[
    \mathrm{Var}(ax+b) = a^2 \mathrm{Var}(x)
\]
\paragraph{}To help model this, suppose we had a dataset of midterm scores $S$. Suppose we added
two points to each midterm score.
\begin{itemize}
    \item The average, $\mu$ of our scores shifts forward by two points.
    \item The variance does not move.
\end{itemize}
\paragraph{}So
\[
    \mathrm{Var}(x+b) = \mathrm{Var}(x)
.\]
\paragraph{}Suppose instead we multiply all the scores by 5, 
\begin{itemize}
   \item Then all our scores get stretched by five fold, with equal probability measure. 
   \item Our average shifts to 5 times what it was previously.
   \item If our distances have stretched by five fold, then our variance has increased by 25.
   \item Our standard deviation has shifted by five fold.
\end{itemize}
\end{document}



