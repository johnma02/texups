\documentclass[letterpaper,12pt]{article}
\setlength{\headheight}{15pt}
\setlength{\marginparwidth}{0pt}
\setlength{\marginparsep}{0pt} % width of space between body text and margin notes
\setlength{\evensidemargin}{0.125in} % Adds 1/8 in. to binding side of all 
% even-numbered pages when the "twoside" printing option is selected
\setlength{\oddsidemargin}{0.125in} % Adds 1/8 in. to the left of all pages when "oneside" printing is selected, and to the left of all odd-numbered pages when "twoside" printing is selected
\setlength{\textwidth}{6.375in} % assuming US letter paper (8.5 in. x 11 in.) and side margins as above
\raggedbottom
\setlength{\parskip}{\medskipamount}

\usepackage{amsmath, amsthm, amssymb, fancyhdr, enumitem, tikz, pgfplots}
\pagestyle{fancy}
\lhead{MATH350 --- HW4}
\begin{document}
\paragraph{Normal Variables}
\paragraph{}Suppose $Z \sim N(0,1)$. Let $\mu \in \mathbb{R}$, $\sigma > 0$.

\paragraph{}Let $X = \sigma Z + \mu$.

\begin{align*}
    E(x) &= E(\sigma Z + \mu)\\
         &= \sigma E(z)+ \mu \\ 
         &= \mu
\end{align*}
\begin{align*}
    \mathrm{Var}(x) &= \mathrm{Var}(\sigma Z + \mu)\\
                    &= \sigma^2 \mathrm{Var}(z)\\
                    &= \sigma^2
\end{align*}
\begin{align*}
    F_X(x) &= P(X \le x)\\
           &= P(\sigma Z + \mu \le x)\\
           &= P(\sigma Z \le x - \mu)\\
           &= P(Z \le \frac{x-\mu}{\sigma})\\
           &= \Phi(\frac{x-\mu}{\sigma})\\
\end{align*}
\begin{align*}
    f_X(x) &= \frac{d}{dx} F_X(x)\\
           &= \frac{d}{dx} \Phi(\frac{x-\mu}{\sigma})\\
           &= \frac{1}{\sigma} \phi(\frac{x-\mu}{\sigma})\\
           &= \frac{1}{\sigma} \frac{1}{\sqrt{2\pi}}e^{-\frac{(\frac{x-\mu}{\sigma})^2}{2}}
\end{align*}
\paragraph{Def:}A random variable $X$ has $N(\mu, \sigma^2)$ distribution if its density is

\[
    f(x) = \frac{1}{\sqrt{2\pi\sigma^2}}e^{-\frac{(\frac{x-\mu}{\sigma})^2}{2}}
\]
\paragraph{}$X$ has mean $\mu$ and variance $\sigma^2$.
\paragraph{}A similar argument can be used to show that if $X \sim N(\mu, \sigma^2)$, then
$X - \frac{\mu}{\sigma} \sim N(0,1)$.
\begin{align*}
    X &= \sigma Z + \mu \\
    \to z &= \frac{x-\mu}{\sigma}
\end{align*}
\end{document}



