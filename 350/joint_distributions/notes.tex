\documentclass[letterpaper,12pt]{article}
\setlength{\headheight}{15pt}
\setlength{\marginparwidth}{0pt}
\setlength{\marginparsep}{0pt} % width of space between body text and margin notes
\setlength{\evensidemargin}{0.125in} % Adds 1/8 in. to binding side of all 
% even-numbered pages when the "twoside" printing option is selected
\setlength{\oddsidemargin}{0.125in} % Adds 1/8 in. to the left of all pages when "oneside" printing is selected, and to the left of all odd-numbered pages when "twoside" printing is selected
\setlength{\textwidth}{6.375in} % assuming US letter paper (8.5 in. x 11 in.) and side margins as above
\raggedbottom
\setlength{\parskip}{\medskipamount}


\usepackage{amsmath, amsthm, amssymb, fancyhdr, enumitem, tikz, pgfplots}

\pgfplotsset{compat=1.18}

\pagestyle{fancy}
\lhead{MATH350 --- Joint Distributions}
\begin{document}
\paragraph{Example:} We have an experiment where we flip a coin 3 times. $x$ is equal to the number of tails
on the first flip. $y$ is equal to the total number of tails observed.

$$x \in \{0,1\}$$
$$y \in \{0,1,2,3\}$$


Suppose we also had 
$$z = \begin{cases}
        1 & \text{We have an even number of heads}\\
        0 & \text{We have an odd number of heads}\\
    \end{cases}
$$


\begin{center}
\begin{tabular}{|| c || c | c | c | c ||} 
 \hline
 \hline
 x \textbackslash \, y & 0 & 1 & 2 & 3\\ [0.5ex] 

 \hline\hline
 0  & .125 & .25 & .125 & 0 \\ 
 \hline
1 & 0 & .125 & .25 & .125 \\
 \hline
 \hline
\end{tabular}
\end{center}


In general, suppose $x$ is discrete taking values $x_1, x_2, \ldots, x_k$. $y$ is also discrete
taking values $y_1, y_2, \ldots, y_k$.

\end{document}



