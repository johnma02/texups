\documentclass{article}

\usepackage{amsmath, amsthm, amssymb, fancyhdr}
\pagestyle{fancy}
\lhead{Example from 10/10/2022}
\begin{document}
\paragraph{}$P(D) = 10\%$.
\paragraph{}A test for a disease is positive 99\% of the time if a person has the disease. 

\paragraph{}But if they don't have the disease, they test positive 2\% of the time.

\paragraph{}We want $P(D|P)$. 
\begin{equation}
P(P|D) = .99
\end{equation}
\begin{equation}
    P(P|D^c) = .02
\end{equation}
\begin{align*}
    P(D|P) &= \frac{P(P|D)\cdot P(D)}{P(P|D)\cdot P(D)+P(P|D^c)\cdot P(D^c)}\\
           &= \frac{.99 \cdot .10}{.99 \cdot .10 + .02 * .90}\\
           &= .846
\end{align*}
\end{document}


