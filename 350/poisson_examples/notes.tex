\documentclass[letterpaper,12pt]{article}
\setlength{\headheight}{15pt}
\setlength{\marginparwidth}{0pt}
\setlength{\marginparsep}{0pt} % width of space between body text and margin notes
\setlength{\evensidemargin}{0.125in} % Adds 1/8 in. to binding side of all 
% even-numbered pages when the "twoside" printing option is selected
\setlength{\oddsidemargin}{0.125in} % Adds 1/8 in. to the left of all pages when "oneside" printing is selected, and to the left of all odd-numbered pages when "twoside" printing is selected
\setlength{\textwidth}{6.375in} % assuming US letter paper (8.5 in. x 11 in.) and side margins as above
\raggedbottom
\setlength{\parskip}{\medskipamount}


\usepackage{amsmath, amsthm, amssymb, fancyhdr, enumitem, tikz, pgfplots}

\pgfplotsset{compat=1.18}

\pagestyle{fancy}
\lhead{MATH350 --- Lecture 11-30-22}
\begin{document}
\paragraph{Example:} $S \sim \mathrm{Bin}(40,\frac{1}{2})$.


\textbf{Check $np(1-p)$} 
\[
    np(1-p) = 40 \cdot \frac{1}{2} \cdot \frac{1}{2} = 10
\]


\textbf{Check $np^2$}
\[
    np^2 = 40 \cdot \frac{1}{4} = 10
\]


Our bounds indicate that we should use the normal approximation.


\textbf{Poisson distributions are good for modeling rare events.} 


Assume a random variable $x$ counts occurances of events that are infrequent and not strongly
dependent.
\begin{itemize}
    \item Earthquakes
    \item Arrivals of customers in a line
\end{itemize}


The distributions of $x$ can be approximated by a Poisson($\lambda$) varable where
$\lambda = E(x)$.


\textbf{i.e} $P(x = k)$ is close to $$\frac{e^{-\lambda}\lambda^k}{k!}$$


\paragraph{Example:}A factory experiences 3 accidents a month, on average. What is the probability
that there are 2 accidents in January.


\begin{enumerate}
    \item Assume that the number of accidents in a month are small.
    \item Assume that accidents occur independently.
\end{enumerate}


Then we may conclude that the number of accidents in a month can be modeled by Poisson(3).


\begin{align*}
P(\text{2 accidents in a month}) &= \frac{e^{-3}\cdot 3^{2}}{2!}\\
                                 &= 0.24
\end{align*}


\paragraph{Example:} A proofreader notices that a randomly chosen page in a manuscript has no typos with
chance 0.9. Estimate the probability that a randomly chosen page has 2 typos.

\begin{enumerate}
    \item Is a typo infrequent? Yes.
    \item Are typos independent? We assume they are \textbf{somewhat} independent.
\end{enumerate}


With these assumptions, we let $x$ equal a binomial random variable. Suppose we approximate $x$ by a 
Poisson($\lambda$).


To find the mean of this Poisson, we should find the mean of $x$. We have

\begin{align*}
    P(x = 0) &= 0.9\\
    0.9 &= \frac{e^{-\lambda}\lambda^0}{0!}\\
    -\mathrm{log}(0.9)    &= \lambda\\
                          &= .10536
\end{align*}

Then $x \sim \mathrm{Poisson}(.10536)$
\begin{align*}
    P(x = 2) &= \frac{e^{-\lambda}\lambda^2}{2!}\\
             &= \frac{0.9 \cdot (.10536)^2}{2!}\\
             &= .005
\end{align*}
\end{document}



