\documentclass{article}

\usepackage{amsmath, amsthm, amssymb, fancyhdr}
\pagestyle{fancy}
\lhead{MATH350 - Lecture 10/24/2022}
\begin{document}

\paragraph{Probability Density Function:} A random variable $x$ has a density if
we have a function $f(x)$ such that 
\[
    P(x \le b) = \int_{-\infty}^{b} f(x)\,dx 
\]
\paragraph{}For a uniform probability measure, we find that the density equals
\[ f(x) = \begin{cases}
    0 & x < 0 \\
    1 & x \in [0,1] \\
    0 & x > 1
    \end{cases}
\]
\paragraph{}We also know that 
\[
    P(a \le x \le b) = \int_a^b f(x)\, dx 
\]
\paragraph{}Discrete random variables do not have density.

\paragraph{Ex:}
\[ x = \begin{cases}
    0 & p = \frac{1}{3} \\
    1 & p = \frac{2}{3}
    \end{cases}
\]

\paragraph{}We can try to construct a density function.

\[
    P(x \le 0) = \frac{1}{3} 
\]

\paragraph{}We want
\[
    \int_{0}^{0} f(x)\,dx = \frac{1}{3}
\]
\paragraph{}This is a contradiction, because we know that the definite integral of the same point is equal to zero.

\paragraph{}For continuous random variables, we know that 
\[
  P(x = a) = 0 
\]
\paragraph{}which allows us to have density functions.

\paragraph{Properties of Density Functions:}Suppose $x$ is a random variable with density $f(x)$.
\begin{enumerate}
    \item
    \begin{align*}
        P(a\le x < b) &= P(a \le x \le b) - P(x = b)\\
                      &= \int_a^b f(x)\,dx - 0 \\
                      &= P(a \le x \le b)
    \end{align*}
    \paragraph{}In general, excluding or including one endpoint makes no difference on the probability.
    \item 
        \begin{align*}
            P(-\infty < x < \infty) &= \int_{-\infty}^{\infty} f(x) \,dx\\
            1 &= \int_{-\infty}^{\infty} f(x) \,dx
        \end{align*}
        \paragraph{Ex:}
        \[ f(x) = \begin{cases}
            \frac{1}{x^2} & x \ge 1 \\
            0 & x < 1 \\
            \end{cases}
        \]
        \paragraph{}We seek to prove (a). $f(x)$ is non-negative, and (b). f(x)'s integral from
        $-\infty$ to $\infty$ is one.
        \paragraph{(a).} Yes, f(x) is non-negative at all points.
        \paragraph{(b).}

        \begin{align*}
            \int_{-\infty}^{\infty} f(x)\,dx &= 
            f(x) = \begin{cases}
                \int_{-\infty}^{1} \frac{1}{x^2}& x \ge 1 \\
                0 & x < 1
                \end{cases}\\
                                             &= \frac{-1}{\infty} + 1\\
                                             &= 1
        \end{align*}
        \paragraph{Ex:} 
        \[ f_2(x) = \begin{cases}
            b\sqrt{a^2 - x^2} & \lvert x \rvert \le a \\
            0 & \lvert x \rvert > a
            \end{cases}
        \]
        \paragraph{} Assuming $a > 0$.
        \begin{align*}
            y &= b\sqrt{a^2 - x^2} \\
            y^2  &= b^2(a^2 - x^2)
        \end{align*}
        \paragraph{}Then we want 
        \begin{align*}
        \int_{-\infty}^{\infty} f_2(x) \, dx &= \int_{-a}^{a} b \sqrt{a^2-x^2}\,dx\\
                                             &= b \int_{-a}^{a} \sqrt{a^2-x^2}\,dx\\
        \end{align*}
        \begin{align*}
            \text{Let } x &= a\sin\theta\\
            dx &= a\cos\theta \, d\theta\\
            \sin\theta&= -1
        \end{align*}
        \begin{align*}
            b \int_{-a}^{a} \sqrt{a^2-x^2}\,dx &= b\int_{-\frac{\pi}{2}}^{\frac{\pi}{2}}\sqrt{a^2 - a^2\sin^2\theta} a\cos\theta\,d\theta\\
                                               &= b \int_{-\frac{\pi}{2}}^{\frac{\pi}{2}}a^2\cos^2\,d\theta\\
                                               &= ba^2 \int_{-\frac{\pi}{2}}^{\frac{\pi}{2}} \cos^2\theta\,d\theta\\
                                               &= ba^2 \int_{-\frac{\pi}{2}}^{\frac{\pi}{2}} \frac{1+\cos 2\theta}{2}\,d\theta\\
                                               &= ba^2 \bigg(\frac{1}{2}\theta + \frac{1}{2}\frac{\sin 2\theta}{2}\bigg)\big|_{\frac{-\pi}{2}}^{\frac{\pi}{2}}\\
                                               &= ba^2 \frac{\pi}{2}
        \end{align*}
        \paragraph{}Thus,
        \[
            b = \frac{2}{\pi\cdot a^2} 
        \]
\end{enumerate}
\paragraph{Uniform Random Variables:} A random variable $x$ is $x \sim U[a,b]$ if and only if it has density
\[ f(x) = \begin{cases}
    c & a \le x \le b \\
    0 & \text{otherwise}
    \end{cases}
\]
\begin{align*}
    (b\cdot c)-(a\cdot c) &= 1\\
    c(b-a) &= 1\\
           &= \frac{1}{b-a}
\end{align*}
\paragraph{}Thus, more formally, the density is defined as
    \[ f(x) = \begin{cases}
        \frac{1}{b-a} & a \le x \le b \\
    0 & \text{otherwise}
    \end{cases}
\]
\paragraph{Ex:}Let $y \sim U[-2,5]$, what is $P(\lvert y \rvert \ge 1)$?
\begin{align*}
    P(\lvert y \rvert \ge 1) &= P(\{y\ge 1\} \cup \{y \le -1\})\\
                             &= P(y \ge 1) + P(y \le -1)\\
                             &= \int_{1}^{\infty} f(x)\,dx + \int_{-\infty}^{-1}f(x)\,dx\\
                             &= \int_{1}^{5} f(x)\,dx + \int_{-2}^{-1}f(x)\,dx\\
                             &= \frac{1}{7}x \big|_{1}^{5} + \frac{1}{7}x \big|_{-2}^{-1}\\
                             &= \frac{5}{7}
\end{align*}
\paragraph{Cumulative Distribution Functions:}There are random variables that do not have density nor are fully discrete.
\paragraph{}We may have a random variable $x$ which may take values $\in [0,1]$, but an additional probability
to take values outside that range, such as $\frac{3}{2}$ with probability $\frac{1}{4}$.

\paragraph{}The cumulative distribution function (cdf) of a random variable $x$ is the function on ($-\infty, \infty$) given by
\[
    F(t) = P(x\le t), \text{for } t \in (-\infty, \infty) 
\]
\paragraph{}All random variables have a cdf.
\paragraph{Ex:} Roll a die,
\[ x = \begin{cases}
    1 & \text{die} \ge 5 \\
    0 & \text{otherwise}\\
    \end{cases}
\]
\paragraph{}This is a Bernoulli random variable, with pmf
\[ f(x) = \begin{cases}
    \frac{1}{3} & x = 1 \\
    \frac{2}{3}& x = 0 \\
    \end{cases}
\]
\paragraph{}and cdf
\[ F(t) = \begin{cases}
    0 & t < 0 \\
    \frac{2}{3} & t = 0 \\
    \frac{2}{3} & 0 \ge t < 1 \\
    1 & t = 1\\
    1 & t \ge 1
    \end{cases}
\]
\paragraph{Ex:} Roll a die
\[ x = \begin{cases}
    \frac{1}{6} & x = 1 \\
    \frac{1}{6} & x = 1 \\
    \,\vdots & \\
    \frac{1}{6} & x = 6
    \end{cases}
\]
\paragraph{}The cdf is equal to
\[ F(t) = \begin{cases}
    0 & t < 0 \\
    \frac{1}{6} & t = 1 \\
    \frac{2}{6} & t = 2 \\
    \,\vdots & \\
    1 & t = 6\\
    1 & t > 6 
    \end{cases}
\]
\end{document}

