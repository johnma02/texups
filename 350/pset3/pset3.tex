\documentclass{article}
\usepackage{amsmath, amsthm, amssymb, fancyhdr}
\pagestyle{fancy}
\lhead{MATH350 - HW3}
\begin{document}
\begin{enumerate}
    \item A fair coin is flipped three times. What is the probability that the second flip is tails
given that there is at most one tail among the other flips?
    
    \paragraph{Ans:}We want to find the following probability measure:
    \begin{align*}
        P(\text{Flip 2 is tails $|$ There is at most one other tail between Flip 1, Flip 3})
    \end{align*}

    \paragraph{}The general formula for conditional probability is
    \begin{align*}
        P(A \vert B) = \frac{P(A \cap B)}{P(B)}
    \end{align*}
    \paragraph{}Let's find the probability of B first. Our sample space is all the possible outcomes of flipping a coin
    three times. The size of our sample space is $2^3$. 


    \paragraph{}Let's consider the possible arrangements of flips 1, 3. We have $2^2$ possibilities, and we are
    interested in finding the number of arrangements where we flip at most one tails. 
    We can intuitively enumerate this by finding the number of arrangements where
    we flip two tails, and subtracting 1 minus it. 

    \paragraph{}There is exactly one arrangement where we flip two tails, therefore
    \begin{align*}
        P(B) &= 1 - \frac{1}{4}\\
             &= \frac{3}{4}
    \end{align*}
    \paragraph{}Finally, the probability that we flip a tails on our second flip and flip only one tail
    among flips 1, 3 is $\frac{1}{2}\cdot \frac{3}{4}$. Therefore
    \begin{align*}
        P(A \vert B) &= \frac{\frac{1}{2} \cdot \frac{3}{4}}{\frac{3}{4}}\\
                     &= \frac{1}{2}
    \end{align*}

    \paragraph{}This result makes sense, because there is no dependence between fair coin flips.
    
    \item Suppose $P(A|B) = 0.6$ and $P(B) = 0.5$. Find $P(A^c \cap B)$.
    \paragraph{Ans:}
    \begin{equation}
        P(A|B) = \frac{P(A\cap B)}{P(B)}
    \end{equation}
    \paragraph{}Using (1)
    \begin{align*}
        P(A\cap B) &= P(A|B) \cdot P(B)\\
                   &= 0.6 \cdot 0.5 \\
                   &= 0.3
    \end{align*}
    \begin{equation}
        P(B) = P(B \cap A) + P(B \cap A^c)
    \end{equation}
    \paragraph{}Using (2)
    \begin{align*}
        0.5 &= 0.3 + P(A^c \cap B)\\
        P(A^c \cap B) &= 0.2
    \end{align*}
    \item Two 6-sided dice are rolled. Find the conditional probability that at least one of the
numbers is even given that the sum is 8.

    \paragraph{Ans:} We want to find
    \begin{align*}
        P(\text{At least one of two dice rolls is even \textbar{}  Sum of two dice rolls is 8})
    \end{align*}
    \paragraph{}The first thing we should consider is the possible outcomes of our experiment which result
    in our desired event.
    
    \paragraph{}The following are the only possible dice rolls of a six-sided die which sum to 8: (2,6), (3,5), (4,4), (5,3), (6,2).

    \paragraph{}We only want to consider the dice rolls with at least one even number. We also take note
    that for two numbers to sum to an even number, either both numbers are even, or both numbers are odd.

    \paragraph{}Of the 5 possible outcomes resulting in a sum of 8, 3 have even addends. Our
    sample space has $6^2$ elements, therefore
    \begin{align*}
        P(A \cap B) &= \frac{3}{36}
    \end{align*}

    \paragraph{}Finally,
    \begin{align*}
        P(A | B) &= \frac{\frac{3}{36}}{\frac{5}{36}}\\
                 &= \frac{3}{36} \cdot \frac{36}{5} \\
                 &= \frac{3}{5}
    \end{align*}

    \item Urn 1 contains balls labeled 1 and 2. Urn 2 contains balls labeled 2, 3, 4, 5. Choose one
    urn randomly so that Urn 1 is chosen with chance 1/5 and Urn 2 is chosen with chance
    4/5. After choosing an urn, we sample a ball from it uniformly at random. Suppose ball
    2 was chosen. What is the probability it came from Urn 2?

    \paragraph{Ans:}We seek to calculate
    \begin{align*}
        P(\text{Urn 2 was chosen \textbar{} Ball 2 was chosen})
    \end{align*}
    
    \paragraph{}We can find the likelihood of ball 2 being chosen as follows: We choose one of
two urns, and then we choose ball 2 in that urn. These probabilities are disjoint, so we get the following:
$\frac{1}{2}\cdot \frac{1}{2} + \frac{1}{2}\cdot \frac{1}{4} = \frac{3}{8}$. 
    \paragraph{}We also observe that the probability that urn 2 is chosen and ball 2 is chosen is
    $\frac{1}{2} \cdot \frac{1}{4} = \frac{1}{8}$.

    Therefore, 
    \begin{align*}
        P(A|B) &= \frac{P(A \cap B)}{P(B)}\\
               &= \frac{\frac{1}{8}}{\frac{3}{8}}\\
               &= \frac{1}{3}
    \end{align*}

    
    \item An insurance company has two types of customers: Careful and Reckless. During one
year, a Careful customer has an accident with probability 0.01 while a Reckless customer
has one with chance 0.04. 80\% of customers are Careful and 20\% are Reckless. Suppose a
randomly chosen customer has an accident. What is the chance it is a Careful customer?

    \paragraph{Ans: }We seek to calculate
    \begin{align*}
        P(\text{A customer is a Careful customer \textbar{} A customer has an accident})
    \end{align*}

    \paragraph{}Suppose we had 100 insured customers. 80 of these customers are careful, and 
    20 are reckless. Of the 80 careful customers, 0.8 of these customers had an accident, and of
    the 20 reckless customers, 0.8 of these customers had an accident. In total, 1.6 people had
    accidents, therefore, the odds of an insured customer having an accident is .16.

    \paragraph{}We already know the likelihood of a careful customer having an accident is 
    .01, therefore.
    
    \begin{align*}
        P(A|B) &= \frac{P(A \cap B)}{P(B)}\\
               &= \frac{.01}{.16}\\
               &= .0625
    \end{align*}

    \item Suppose $P(A) = 0.25$, $P(B) = 0.65$. If $A$ and $B$ are disjoint, what is $P(A \cup B)$?
        If $A$ and $B$ are independent what is $P(A \cup B)$?

    \paragraph{}If $A$ and $B$ are disjoint, then $P(A \cup B) = P(A) + P(B) = 0.90$.
    \paragraph{}If $A$ and $B$ are independent, then
    \begin{align*}
        P(A \cap B) &= P(A|B)\cdot P(B) \\
                    &= P(A) \cdot P(B)\\
                    &= .1625
    \end{align*}

    \item A fair coin is flipped 3 times. For $i = 1, 2, 3$, let $A_i$ be the event that among the first $i$
coin flips, there are an odd number of heads. Are the events $A_1$, $A_2$, $A_3$ independent or
not?
    \paragraph{Ans:}To find if the $A_i$ are independent, we must verify that the probability of all the possible
    intersections of events is equal to their composite probabilities multiplied.

    \paragraph{} $P(A_1) = \frac{1}{2}$, as we want our first and only flip to be a heads.
    \paragraph{} $P(A_2) = \frac{1}{2}$, as want either our first or our second flip to be heads. (four total
    configurations, two are successes)
    \paragraph{} $P(A_3) = \frac{1}{2}$, as now we want either only one head or all three flips to be heads. (eight
    total configurations, four are successes).

    \paragraph{} $P(A_1 \cap A_2) = \frac{1}{4}$. We know that our first flip must be heads, therefore, our
    second flip must be tails. There are four configurations for two flips, and only one satisfies this.

    \paragraph{} $P(A_1 \cap A_3) = \frac{1}{4}$. We know that our first flip must be heads, and therefore
    we must either have three heads by flip three, or one head by flip three. There are eight total configurations
    for three flips, and only two of these satisfies this.

    \paragraph{} $P(A_2 \cap A_3) = \frac{1}{4}$. We know that either our second flip was a heads and our first
    was a tails or vice versa. Thus, by the third flip, we must have only one heads, and thus the third flip must be 
    a tails. There are eight configurations for three flips, and only two satisfy this.

    \paragraph{}We have verified that that the probability of our pairwise intersections is equal to their 
    products. ($\frac{1}{2} \cdot \frac{1}{2} = \frac{1}{4}$).

    \paragraph{}Finally, $P(A_1 \cap A_2 \cap A_3) = \frac{1}{8}$. We know that the first flip was heads, and that the second
    flip was a tails. Therefore, the third flip must also be a tails. There are eight configurations for three
    flips and only this one scenario satisfies this.

    \paragraph{}With this, we verify that $P(A_1 \cap A_2 \cap A_3) = \frac{1}{2} \cdot \frac{1}{2} \cdot \frac{1}{2}$, 

    which is the product of composite probabilities. Therefore, the events $A_1, A_2, A_3$ are independent.
    \item We choose a number from the set $\{10, 11, 12, \ldots , 99\}$ uniformly at random.

    \paragraph{} (a). Let $X$ be the first digit and let $Y$ be the second digit. Show that $X$ and $Y$ are
independent.

    \paragraph{} (b.) Let $Z$ be the sum of the digits. Argue that $X$ and $Z$ are not independent.

\item We choose one of the words in the following sentence at random and the choose one of
the letters of that word uniformly at random (within the word).
SOME DOGS ARE YELLOW
\paragraph{}(a) Find the probability that the chosen letter is E.

\paragraph{Ans:} Considering that all words and letters are chosen uniformly at random, we have a
$\frac{1}{4}$ probability to choose any given word, and from there, we should consider the ratio of the number
of 'E's in the word, and the total number of letters in the word.
\begin{itemize}
    \item "SOME": Four letters, one 'E': $\frac{1}{4}$.
    \item "DOGS": Four letters, no 'E's: 0.
    \item "ARE": Three letters, one 'E': $\frac{1}{3}$.
    \item "YELLOW": Six letters, one 'E': $\frac{1}{6}$.
\end{itemize}

Then we can calculate the probability through summation. 
\begin{align*}
    P(\text{Letter is E}) &= \frac{1}{4} \cdot (\frac{1}{4}+0+\frac{1}{3}+\frac{1}{6})\\
                          &= \frac{1}{4} \cdot \frac{3}{4}\\
                          &= \frac{3}{16}
\end{align*}
We can verify this using a simple Python script.
\begin{verbatim}
import random
success=0
for i in range(0, 1000000):

    sent = ["SOME", "DOGS", "ARE", "YELLOW"]

    word = random.randint(0, 3)

    letter = random.randint(0, len(sent[word])-1)

    if sent[word][letter] == 'E':
        success+=1
print(success/1000000)

>> 0.187881
\end{verbatim}
\paragraph{}($\frac{3}{16} = .1875$)
\paragraph{}(b) Let $X$ be the length of the chosen word. Find the pmf of $X$.
\paragraph{Ans:} There are three lengths the words may take: 3, 4 and 6. 
\[ f(x) = \begin{cases}
    \frac{1}{4} & x = 3 \\
    \frac{1}{2}  & x = 4\\
    \frac{1}{4}   & x = 6\\
    \end{cases}
\]
\paragraph{}(c) For each possible value $k$ of $X$, find the conditional probability $P(X = k|X > 3)$.
\paragraph{Ans:}Given that the length of our chosen word is greater than three, we have
\[ f(k) = \begin{cases}
    0 & k = 3 \\
    \frac{2}{3} & k = 4 \\
    \frac{1}{3} & k = 6 \\
    \end{cases}
\]
\paragraph{}(d) Find the conditional probability $P(\text{The chosen letter is E} |X > 3)$.
\paragraph{Ans:}Given that the length of our chosen word is greater than three, we have three possible words:
"SOME", "DOGS", "YELLOW". Following the same process as shown in (a)., and using the result from (c).
\begin{align*}
    P(\text{The chosen letter is E} |X > 3) &= \frac{1}{3} \cdot (\frac{1}{4}+0+\frac{1}{6})\\
                                            &= \frac{1}{3} \cdot \frac{10}{24}\\
                                            &= \frac{10}{72}
\end{align*}
We use a similar Python script to verify this.
\begin{verbatim}
>> 0.138886
\end{verbatim}

\paragraph{}(e) Given that the chosen letter is E, what is the probability that the chosen word was
YELLOW?

\end{enumerate}



\end{document}

