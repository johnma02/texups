\documentclass[letterpaper,12pt]{article}
\setlength{\headheight}{15pt}
\setlength{\marginparwidth}{0pt}
\setlength{\marginparsep}{0pt} % width of space between body text and margin notes
\setlength{\evensidemargin}{0.125in} % Adds 1/8 in. to binding side of all 
% even-numbered pages when the "twoside" printing option is selected
\setlength{\oddsidemargin}{0.125in} % Adds 1/8 in. to the left of all pages when "oneside" printing is selected, and to the left of all odd-numbered pages when "twoside" printing is selected
\setlength{\textwidth}{6.375in} % assuming US letter paper (8.5 in. x 11 in.) and side margins as above
\raggedbottom
\setlength{\parskip}{\medskipamount}


\usepackage{amsmath, amsthm, amssymb, fancyhdr, enumitem, tikz, pgfplots}

\pgfplotsset{compat=1.18}

\pagestyle{fancy}
\lhead{MATH302 --- Lecture 11-28-22}
\begin{document}
\paragraph{Locally Linear Systems}

\paragraph{Theorem 9.3.1}The critical point $x = 0$ of the linear system 
\begin{equation}
    x^{\prime} = Ax.
\end{equation}

\begin{enumerate}
    \item The system is asymptotically stable if the eigenvalues $r_1, r_2$ are real and negative, or have
        negative real parts.
    \item The system is stable but not asymptotically stable if $r_1, r_2$ are purely 
        imaginary
    \item The system is unstable if $r_1, r_2$ are real and exclusively one is positive or if 
        they both have real positive parts.
\end{enumerate}

Consider a nonlinear two-dimensional system 
\[
    x^{\prime} = f(x)
\]


If $x = \begin{bmatrix}
    x\\y
\end{bmatrix}$, and $f(x) = \begin{bmatrix}
    F(x,y)\\G(x,y)
\end{bmatrix}$, then 


$$
    \begin{bmatrix}
        x^{\prime}\\ 
        y^{\prime}
    \end{bmatrix} = 
    \begin{bmatrix}
    F(x,y)\\G(x,y)
    \end{bmatrix}
$$


Suppose that $x^{\prime} = Ax + g(x)$ and that $x=0$ is its isolated critical point 
(There is some circle about the origin within which there are no other critical points).


Additionally, assume that the $ \mathrm{det}(A) \ne 0$, which implies $x=0$ is an isolated critical point
of the system $x^{\prime} = Ax$.


For the nonlinear system to be "close" to the linear system, we assume that $g(x)$ is small, i.e. $g(x)$ 
satisfies the following:
\begin{enumerate}
    \item The components of $g(x)$ have continuous first partial derivatives.
    \item It satisfies the following limit condition
        $$ \frac{\lVert g(x) \rVert}{\lVert x \rVert} \to 0$$
\end{enumerate}


Such a system is caalled a locally linear system in the neighborhood of the critical point
$x = 0$.


If $x = \begin{bmatrix}
    x\\y
\end{bmatrix}$, $\lVert x \rVert = \sqrt{x^2 + y^2} = r$, and


$g(x) = \begin{bmatrix}
    g_1(x,y)\\g_2(x,y)
\end{bmatrix}$, $\lVert g(x) \rVert = \sqrt{(g_1(x,y))^2 + (g_2(x,y))^2}$, then our conditions
are satisfied if and only if 
\[
    \frac{g_1(r\cos(\theta), r\sin(\theta))}{r} \to 0, 
    \frac{g_2(r\cos(\theta), r\sin(\theta))}{r} \to 0, 
\]
as $r\to 0$ for all $0 \leq \theta \leq 2\pi$.


\paragraph{Example 1:} Determine whether the system

\[
    \begin{bmatrix}
        x\\y
    \end{bmatrix}^{\prime}
    =
    \begin{bmatrix}
        1 & 0\\
        0 & 0.5
    \end{bmatrix}
    \begin{bmatrix}
        x\\y
    \end{bmatrix}
    +
    \begin{bmatrix}
        -x^2 - xy \\
        -0.75x - 0.25x^2
    \end{bmatrix}
\]
is locally linear in the neighborhood of (0,0).

\paragraph{Solution:}(0,0) is a critical point of our system. $\det(A) \ne 0$.


The other critical points are (0,2), (1,0), ($\frac{1}{2}, \frac{1}{2}) \to (0,0)$ is an
isolated critical point.


\begin{align*}
    \frac{g_1(r\cos\theta, r\sin\theta)}{r} &= -r(\cos^2\theta + \cos\theta\sin\theta)
\end{align*}


This approaches 0 as $r$ approaches 0. We can use the same argument for $\frac{g_2(x,y)}{r}$,
therefore, our system is locally linear near the origin.

\paragraph{Theorem 9.3.2}Nonlinear systems are locally linear in the neighborhood of a critical point
($x_0, y_0$) whenever the functions $F, G$ have continuous partial derivatives up to the second order.
\end{document}



