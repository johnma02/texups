\documentclass{article}

\usepackage{amsmath, amsthm, amssymb, fancyhdr}
\pagestyle{fancy}
\lhead{MATH302: Lecture 10/21/2022}
\begin{document}

\paragraph{Linear Systems of Differential Equations:} A system of differential equations is called linear if each of $F_1, F_2, \ldots, F_n$ is a
linear function of $x_1, x_2, \ldots, x_n$. Otherwise, it is called non-linear.

\paragraph{} We are only considering linearity of the variable $x$.

\[
    x_{1}' = P_{11}(t)x_1 + p_{12}x_2(t) + \ldots + p_{1n}(t) x_n + g_1(t)\\
\]
\[
    x_{2}' = P_{21}(t)x_1 + p_{22}x_2(t) + \ldots + p_{2n}(t) x_n + g_1(t)\\
\]
\[
    \vdots\\
    \]
    \[
    x_{n}' = P_{n1}(t)x_1 + p_{n2}x_2(t) + \ldots + p_{nn}(t) x_n + g_1(t)\\
\]

\[
\begin{bmatrix}
    x_1' \\
    x_2' \\
    \vdots \\
    x_n'
\end{bmatrix}
=
\begin{bmatrix}
    p_{11}(t) & p_{12}(t) & \ldots & p_{1n}(t) \\
    p_{21}(t) & p_{22}(t) & \ldots & p_{2n}(t) \\
    \vdots &  & \ddots & \vdots \\ 
    p_{n1}(t) & p_{n2}(t) & \ldots & p_{nn}(t) \\
\end{bmatrix}
\begin{bmatrix}
    x_1 \\
    x_2 \\
    \vdots \\
    x_n
\end{bmatrix}
\]

\paragraph{Theorem 7.1.2:} If the functions $p_{11}$, $p_{12}$, $\ldots$, $p_{nn}$, $g_1$, $g_2$, $\ldots$,
$g_n$ are continuous on an open interval $I$, $\alpha < t < \beta$, then there exists an unique solution
$x_1 = \phi_1 (t), \ldots , x_n = \phi_n (t)$ of the system that satisfies the initial condition
problem where $t_0$ is any point in $I$ and $x_1^{[0]}, \ldots, x_n^{[0]}$ are any prescribed numbers.

\paragraph{}Then the solution exists throughout the interval $I$.
\paragraph{Linear Algebra Review}
\paragraph{AB}

\[
    \begin{bmatrix}
        1 & -2 & 1 \\ 
        0 & 2 & -1 \\
        2 & 1 & 1 
    \end{bmatrix}
    \cdot
    \begin{bmatrix}
        2 & 1 & -1 \\
        1 & -1 & 0 \\
        2 & -1 & 1
    \end{bmatrix}
    =
    \begin{bmatrix}
        2 & 2 & 0\\
        0 & -1 & -1 \\
        7& 0 & -1 
    \end{bmatrix}
\]
\paragraph{BA}
\[
    \begin{bmatrix}
        2 & 1 & -1 \\
        1 & -1 & 0 \\
        2 & -1 & 1
    \end{bmatrix}
    \cdot 
    \begin{bmatrix}
        1 & -2 & 1 \\ 
        0 & 2 & -1 \\
        2 & 1 & 1 
    \end{bmatrix}
    = 
     \begin{bmatrix}
        0 & -3 & 0 \\ 
        1 & -4 & 2 \\
        4 & -5 & 4 
    \end{bmatrix} 
\]
\paragraph{$AB \ne BA$}

\end{document}

