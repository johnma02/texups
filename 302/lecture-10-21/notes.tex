\documentclass{article}

\usepackage{amsmath, amsthm, amssymb, fancyhdr}
\pagestyle{fancy}
\lhead{MATH302: Lecture 10/21/2022}
\begin{document}

\paragraph{Linear Systems of Differential Equations:} A system of differential equations is called linear if each of $F_1, F_2, \ldots, F_n$ is a
linear function of $x_1, x_2, \ldots, x_n$. Otherwise, it is called non-linear.

\paragraph{} We are only considering linearity of the variable $x$.

\[
    x_{1}' = P_{11}(t)x_1 + p_{12}x_2(t) + \ldots + p_{1n}(t) x_n + g_1(t)\\
\]
\[
    x_{2}' = P_{21}(t)x_1 + p_{22}x_2(t) + \ldots + p_{2n}(t) x_n + g_1(t)\\
\]
\[
    \vdots\\
    \]
    \[
    x_{n}' = P_{n1}(t)x_1 + p_{n2}x_2(t) + \ldots + p_{nn}(t) x_n + g_1(t)\\
\]

\[
\begin{bmatrix}
    x_1' \\
    x_2' \\
    \vdots \\
    x_n'
\end{bmatrix}
=
\begin{bmatrix}
    p_{11}(t) & p_{12}(t) & \ldots & p_{1n}(t) \\
    p_{21}(t) & p_{22}(t) & \ldots & p_{2n}(t) \\
    \vdots &  & \ddots & \vdots \\ 
    p_{n1}(t) & p_{n2}(t) & \ldots & p_{nn}(t) \\
\end{bmatrix}
\begin{bmatrix}
    x_1 \\
    x_2 \\
    \vdots \\
    x_n
\end{bmatrix}
\]

\paragraph{Theorem 7.1.2:} If the functions $p_{11}$, $p_{12}$, $\ldots$, $p_{nn}$, $g_1$, $g_2$, $\ldots$,
$g_n$ are continuous on an open interval $I$, $\alpha < t < \beta$, then there exists an unique solution
$x_1 = \phi_1 (t), \ldots , x_n = \phi_n (t)$ of the system that satisfies the initial condition
problem where $t_0$ is any point in $I$ and $x_1^{[0]}, \ldots, x_n^{[0]}$ are any prescribed numbers.

\paragraph{}Then the solution exists throughout the interval $I$.
\paragraph{Linear Algebra Review}
\paragraph{AB}

\[
    \begin{bmatrix}
        1 & -2 & 1 \\ 
        0 & 2 & -1 \\
        2 & 1 & 1 
    \end{bmatrix}
    \cdot
    \begin{bmatrix}
        2 & 1 & -1 \\
        1 & -1 & 0 \\
        2 & -1 & 1
    \end{bmatrix}
    =
    \begin{bmatrix}
        2 & 2 & 0\\
        0 & -1 & -1 \\
        7& 0 & -1 
    \end{bmatrix}
\]
\paragraph{BA}
\[
    \begin{bmatrix}
        2 & 1 & -1 \\
        1 & -1 & 0 \\
        2 & -1 & 1
    \end{bmatrix}
    \cdot 
    \begin{bmatrix}
        1 & -2 & 1 \\ 
        0 & 2 & -1 \\
        2 & 1 & 1 
    \end{bmatrix}
    = 
     \begin{bmatrix}
        0 & -3 & 0 \\ 
        1 & -4 & 2 \\
        4 & -5 & 4 
    \end{bmatrix} 
\]
\paragraph{$AB \ne BA$}

\paragraph{Orthogonality:} Two vectors $x$, $y$, are orthogonal if and only if ($x$,$y$) = 0.

\paragraph{Ex:} 
\[
  z = 
  \begin{bmatrix}
      1 \\ 0 \\ i
  \end{bmatrix}
\]
\[
  z^Tz = 1 + i^2 = 0
\]
\[
    (z,z) = 
    \begin{bmatrix}
        1 & 0 & i
    \end{bmatrix}
    \cdot
    \begin{bmatrix}
    1 \\ 0 \\ -i
    \end{bmatrix}
    =
    2
\]
\paragraph{Identity Matrix:}
\[
    \begin{bmatrix}
        1 & 0 & 0\\
        0 & 1 & 0\\
        0 & 0 & 1\\
    \end{bmatrix}
\]
\paragraph{Inverse:} $A \in \mathbb{R}^{n x n}$ is called nonsingular or invertible if there exists
another matrix $B$ such that $AB = I$, and $BA = I$. 

\paragraph{} $B$ is the inverse of A, also written $B = A^{-1}$.
\[
AA^{-1} = A^{-1}A = I 
\]

\paragraph{Determinant of a Matrix:} Let $A \in \mathbb{R}^{nxn}$. 

\[
  \begin{bmatrix}
      a_{11} &\ldots  & a_{1n}\\
      \vdots & a_{ij} & \vdots\\
      a_{n1} & \ldots & a_{nn}\\
  \end{bmatrix} 
\]

\[
    \det(A) = c_{11} + c_{12} + \ldots + c_{1n} 
\]
\paragraph{}where $c_{ij}$ is the cofactor.
\paragraph{Ex:}
\begin{align*}
  \begin{vmatrix}
      1 & 2 & -4\\
      2 & 1 & 1\\
      -1 & 3 & -11\\
  \end{vmatrix} 
   &=
   1(-11 - 3) - 2(-22 + 1) -4(6 + 1)\\
   &= -14 + 42 - 28 \\
   &= 0
\end{align*}
\paragraph{Note:} A matrix is called singular if and only if $\det(A) = 0$.

\paragraph{Computing $A^{-1}$:}

\makeatletter
\renewcommand*\env@matrix[1][*\c@MaxMatrixCols c]{%
  \hskip -\arraycolsep
  \let\@ifnextchar\new@ifnextchar
  \array{#1}}
\makeatother

\[
\begin{bmatrix}[ccc|ccc]
    1& -1 & -1 & 1 & 0 & 0\\
    3 & -1 & 2 & 0 & 1 & 0\\
    2 & 2 & 3 & 0 & 0 & 1\\
  \end{bmatrix} 
\]
$R_2 - 3R_1,$
$R_3 - 2R_1$


\[
\begin{bmatrix}[ccc|ccc]
    1& -1 & -1 & 1 & 0 & 0\\
    0 & 2 & 5 & -3 & 1 & 0\\
    0 & 4 & 5 & -2 & 0 & 1\\
  \end{bmatrix} 
\]
$R_1 + \frac{1}{2}R_2,$
$R_3 - 2R_2$

\[
\begin{bmatrix}[ccc|ccc]
    1& 0 & \frac{3}{2} & \frac{-1}{2} & \frac{1}{2} & 0\\
    0 & 2 & 5 & -3 & 1 & 0\\
    0 & 0 & -5 & 4 & -2 & 1\\
  \end{bmatrix} 
\]

$R_1+\frac{3}{10}R_3,$
$R_2 + R_3$

\[
\begin{bmatrix}[ccc|ccc]
    1& 0 & 0 & \frac{7}{10} & \frac{-1}{10} & \frac{3}{10}\\
    0 & 2 & 0 & 1 & -1 & 1\\
    0 & 0 & -5 & 4 & -2 & 1\\
  \end{bmatrix} 
\]

$\frac{1}{2}R_2, \frac{-1}{5}R_3$

\[
\begin{bmatrix}[ccc|ccc]
    1& 0 & 0 & .7 & -.1 & .3\\
    0 & 1 & 0 & .5 & -.5 & .5\\
    0 & 0 & 1 & -.8 & .4 & -.2\\
  \end{bmatrix} 
\]
\[
    A^{-1} =  
    \begin{bmatrix}
        .7 & -.1 & .3\\
        .5 & -.5 & .5\\
        -.8 & .4 & -.2\\
    \end{bmatrix}
\]


\end{document}
