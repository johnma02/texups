\documentclass[letterpaper,12pt]{article}
\setlength{\headheight}{15pt}
\setlength{\marginparwidth}{0pt}
\setlength{\marginparsep}{0pt} % width of space between body text and margin notes
\setlength{\evensidemargin}{0.125in} % Adds 1/8 in. to binding side of all 
% even-numbered pages when the "twoside" printing option is selected
\setlength{\oddsidemargin}{0.125in} % Adds 1/8 in. to the left of all pages when "oneside" printing is selected, and to the left of all odd-numbered pages when "twoside" printing is selected
\setlength{\textwidth}{6.375in} % assuming US letter paper (8.5 in. x 11 in.) and side margins as above
\raggedbottom
\setlength{\parskip}{\medskipamount}

\usepackage{amsmath, amsthm, amssymb, fancyhdr, enumitem, tikz}
\pagestyle{fancy}
\lhead{Lecture --- 11-04-2022}
\begin{document}
\paragraph{Ex:}Find the general solution of
\[
    x^{\prime}
    \begin{bmatrix}
        -3 & \sqrt{2}\\
        \sqrt{2} & -2
    \end{bmatrix}x
\]
\[
    x = \xi e^{rt}
\]
\[
    (A-rI)\xi = 0
\]
\begin{align*}
    \begin{bmatrix}
        -3 - r & \sqrt{2}\\
        \sqrt{2} & -2-r
    \end{bmatrix}\xi
    &= 0\\
    (-3-r)(-2-r)-2 &= 0 \\
    6 + 3r + 2r +r^2 - 2 &= 0\\
    r^2 + 5r +4 &= 0\\
    (r+4)(4+1) &= 0\\
\end{align*}
\paragraph{$r=-1$}
\makeatletter
\renewcommand*\env@matrix[1][*\c@MaxMatrixCols c]{%
\hskip -\arraycolsep
\let\@ifnextchar\new@ifnextchar
\array{#1}}
\makeatother

\[
\begin{bmatrix}[cc|c]
    -2 & \sqrt{2} & 0\\
    \sqrt{2} & -1 & 0\\
\end{bmatrix}
\]
\paragraph{$R_2+\frac{\sqrt{2}}{2}R_1$}
\[
\begin{bmatrix}[cc|c]
    -2 & \sqrt{2} & 0\\
    0 & 0 & 0\\
\end{bmatrix}
\]
$$\xi_1 = \begin{bmatrix}
    1 \\ \sqrt{2}
\end{bmatrix}$$
$$\vdots$$
\[ \xi = \begin{cases}
        x_1 = C_1e^{-t}-C_2\sqrt{2} e^{-4t}\\
        x_2 = C_1\sqrt{2} e^{-t}+C_2 e^{-4t}\\
    \end{cases}
\]
\paragraph{}To solve nonlinear systems, we need to find eigenvalues $r_1,r_2,\ldots,r_n$, by solving 
characteristic polynomial ($A-rI)=0$, and finding the associated eigenvectors.
\paragraph{}There are a couple cases

\begin{enumerate}
    \item All eigenvalues are real and different
    \item Some eigenvectors occur in complex conjugate pairs
    \item Case 2 + and real or complex are repeated.
\end{enumerate}
\paragraph{Case 1:} Eigenvalues $r_1, r_2, \ldots, r_n$ are real and different. Then the associated
eigenvectors $\xi_1, \ldots, \xi_n$ are linearly dependent.

\paragraph{}The corresponding solution of $x^{\prime}= Ax$ is
\[
    x^{(1)} = \xi^{(1)}\cdot e^{r_1 t},\ldots, x^{(1)}(t) = \xi^{(n)}\cdot e^{r_n t}
\]
\paragraph{}The general solution of $x^{\prime} = Ax$ is
\begin{equation}
    x=c_1\xi^{(1)}e^{(r_1 t)} + \ldots + c_n\xi^{(n)}e^{r_n t}
\end{equation}
\paragraph{}If A is real and symmetric,
\begin{enumerate}
    \item All eigenvalues $r_1, \ldots, r_n$ are real.
    \item There is a set of $n$ eigenvectors $\xi^{(1)},\ldots,\xi^{(n)}$ that are
        linearly independent.

    \item The solution will also be written as (1).
\end{enumerate}
\paragraph{Ex:}Find general solution of 
\[
    x^{\prime} = \begin{bmatrix}
         0 & 1 & 1\\
        1 & 0 & 1\\
        1 & 1 & 0\\
    \end{bmatrix}
\]
\paragraph{Eigenvalues, Eigenvectors:}
\paragraph{$\lambda  = -1$}
\[
    x_1= \begin{bmatrix}
        -1\\0\\1
        \end{bmatrix}
    \,
    ,
    \,
    x_2 =
    \begin{bmatrix}
        -1\\1\\0
    \end{bmatrix}
\]
\paragraph{Fundamental Set of Solutions}
\[
    x^{(1)} = \begin{bmatrix}
        1\\1\\1
    \end{bmatrix}e^{2t}
    \,
    ,
    \,
    x^{(2)}= \begin{bmatrix}
        1\\0\\-1
    \end{bmatrix}e^{-t}
    \,
    ,
    \,
    x^{(3)}= \begin{bmatrix}
        0\\1\\-1
    \end{bmatrix}e^{-t}
\]
\paragraph{General Solution:}
\[
    x = C_1 
    \begin{bmatrix}
        1\\1\\1
    \end{bmatrix}e^{2t} + 
    C_2
    \begin{bmatrix}
        1\\0\\-1
    \end{bmatrix}e^{-t}
    +
    \begin{bmatrix}
        0\\1\\-1
    \end{bmatrix}e^{-t}
\]
\paragraph{Ex 1:} Find a fundamental set of real-valued solutions of the system

\[
    x^{\prime}= \begin{bmatrix}
        -\frac{1}{2} & 1\\
        -1 & \frac{1}{2}
    \end{bmatrix}
\]
\begin{align*}
    (-\frac{1}{2}-r)(\frac{1}{2}-r)+1&=0\\
\end{align*}
\end{document}


