\documentclass{article}

\usepackage{amsmath, amsthm, amssymb, fancyhdr, enumitem, tikz}
\pagestyle{fancy}
\lhead{MATH302 --- Eigenvalues, Eigenvectors in ODEs}
\begin{document}


\[
    Ax = y
\]

\paragraph{Eigenvectors:} Using $A$, we may transform $x$ into $y$, where

\[
    y = \lambda x
\]
\paragraph{}and
\begin{align*}
    Ax &= \lambda x\\
       &= \lambda I x\\
    (A-\lambda I)x &= 0
\end{align*}
\paragraph{}There exists non-zero solutions to $(A-\lambda I)x = 0$ if and only if the
characteristic polynomial


\[
    \det (A-\lambda I) = 0
\]

\paragraph{Eigenvalues:} $\lambda$ of $A$ can be real or complex, and each have corresponding
eigenvectors.

\paragraph{Ex:} Find eigenvalues, eigenvectors of the matrix 


\[
    A=
\begin{bmatrix}
    3&-1\\
    4&-2
\end{bmatrix}
\]

\begin{align*}
    \det(A-\lambda I) &= 
\begin{vmatrix}
    3-\lambda&-1\\
    4&-2-\lambda
\end{vmatrix}\\
                      &= (3-\lambda)(-2-\lambda)+4\\
                      &= -6 -3\lambda +2\lambda + \lambda^2 + 4\\
                      &= -2 -1\lambda + \lambda^2 \\
                      &= (\lambda+1)(\lambda-2)
\end{align*}

\makeatletter
\renewcommand*\env@matrix[1][*\c@MaxMatrixCols c]{%
\hskip -\arraycolsep
\let\@ifnextchar\new@ifnextchar
\array{#1}}
\makeatother

\[
    [A-\lambda | 0] =
\begin{bmatrix}[cc|c]
    4 & -1 & 0\\
    4 & -1 & 0\\
\end{bmatrix}
\]
\paragraph{}$R_2 - R_1$
\[
\begin{bmatrix}[cc|c]
    4 & -1 & 0\\
    0 & 0 & 0
\end{bmatrix}
\]
\paragraph{}Therefore

\[
    x = 
    \begin{bmatrix}
        \frac{1}{4}t \\ t
    \end{bmatrix}
\]
\paragraph{}(We will only compute this eigenvector in this example)

\paragraph{Normalization:}For some vector $x$, choose a constant $k$ such that $\lVert k \cdot x\rVert = 1$

\[
    \frac{1}{\sqrt{17}} \cdot
    \begin{bmatrix}
        \frac{1}{4}t \\ t
    \end{bmatrix}
    = 
    \begin{bmatrix}
        \frac{1}{4\sqrt{17}}t \\ \frac{t}{\sqrt{17}}
    \end{bmatrix}
\]

\paragraph{Multiplicity:}If a given eigvenvalue appears $m$ times as a root of the
characteristic polynomial, then that eigenvalue is said to have an algebraic multiplicity of $m$.

\paragraph{}If an eigenvalue has $q$ linearly independent eigenvectors, then it is said to have
geometric multiplicity of $q$.

\paragraph{Properties of Eigenvalues \& Eigenvectors:}

\begin{enumerate}
    \item If each eigenvalue of $A$ is simple (has algebraic multiplicity of 1), then each
        eigenvalue also has geometric multiplicity of 1.
    \item If $A$ has $k$ unique eigenvalues $\lambda_1, \ldots, \lambda_k$, and their corresponding 
        eigenvectors $x_1,\ldots, x_k$. Then $x_1, \ldots, x_k$ are linearly independent.
    \item If $A \in \mathbb{R}^{nxn}$ has one or more repeated eigenvalues, there \emph{may} be fewer
        than $n$ linearly independent eigenvectors associated with $A$.
\end{enumerate}

\paragraph{Ex:} Find the eigenvalues and eigenvectors of 


\[
A=
\begin{bmatrix}
    0 & 1& 1\\
    1& 0 & 1\\
    1 & 1 & 0\\
\end{bmatrix}
\]
\begin{align*}
    \det(A-\lambda I)&=
\begin{bmatrix}
    -\lambda & 1& 1\\
    1& -\lambda & 1\\
    1 & 1 & -\lambda\\
\end{bmatrix}\\
      &= -\lambda(-\lambda\cdot-\lambda - 1)-(-\lambda-1)+(1+\lambda)\\
      &= -\lambda^3 + \lambda + \lambda + 1 + 1 + \lambda\\
      &= -\lambda^3 + 3\lambda + 2\\
      &= (-\lambda^2 + 2\lambda) + (\lambda + 2)\\
      &= (\lambda+1)^2(\lambda-2)
\end{align*}
\paragraph{}$\lambda = 2$

\[
\begin{bmatrix}[ccc|c]
    -2 & 1 & 1 & 0\\
    1 & -2 & 1 & 0\\
    1 & 1 & -2 & 0
\end{bmatrix}
\]
\[
    \vdots
\]
\[
\begin{bmatrix}[ccc|c]
    2 & -1 & -1 & 0\\
    0 & 1 & -1 & 0\\
    0 & 0 & 0 & 0
\end{bmatrix}
\]
\[
    x = \begin{bmatrix}
        t\\t\\t
    \end{bmatrix}
\]
\paragraph{}Let $t = 1$.
\[
    x = \begin{bmatrix}
        1\\1\\1
    \end{bmatrix}
\]
\paragraph{}$\lambda = -1$

\[
\begin{bmatrix}[ccc|c]
    1 & 1 & 1 & 0\\
    1 & 1 & 1 & 0\\
    1 & 1 & 1 & 0
\end{bmatrix}
\]
\[
    \vdots
\]

\[
\begin{bmatrix}[ccc|c]
    1 & 1 & 1 & 0\\
    0 & 0 & 0 & 0\\
    0 & 0 & 0 & 0
\end{bmatrix}
\]
\[
    x = \begin{bmatrix}
        -t-s\\s\\t
    \end{bmatrix}
\]
\[
    x = \begin{bmatrix}
        -1 \\ 0 \\ 1
    \end{bmatrix}t
    + \begin{bmatrix}
        -1 \\ 1 \\ 0
    \end{bmatrix}s
\]
\paragraph{}We extract two eigenvectors. For $x_1$, let $t = 1$, $s = 0$, and for $x_2$, let $t=0$, $s=1$.
\[
    x_1 =
    \begin{bmatrix}
        -1 \\ 0 \\ 1
    \end{bmatrix},
    \,\,
    x_2 = 
    \begin{bmatrix}
        -1 \\ 1 \\ 0
    \end{bmatrix}
\]
\paragraph{Hermitian Matrix:}For $A$ Hermitian, 

\begin{enumerate}
    \item $A^* = A$.
    \item All eigenvalues are real.
    \item There exists $n$ independent eigenvectors.
    \item If $x_1$, $x_2$ are eigenvectors corresponding to $\lambda_1$, $\lambda_2$,
        \[
            (x_1, x_2) = 0
        \]
        Therefore, if all eigenvalues are simple, the associated eigenvectors form an 
        orthogonal set of vectors.
    \item It is possible to choose $m$ eigenvectors that are mutually orthogonal which correspond to
        and eigenvalue of algebraic multiplicity $m$.
\end{enumerate}
\end{document} 

