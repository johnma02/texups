\documentclass{article}

\usepackage{amsmath, amsthm, amssymb, fancyhdr}
\pagestyle{fancy}
\lhead{MATH302 - HW6}
\begin{document}
\begin{verbatim}
Section 3.5 p. 141] #2, 3, 5, 6, 7, 12, 13, 14

Section 3.6 p. 146] #4, 6

Section 3.7 p. 157] # 1, 2, 3, 4
\end{verbatim}
\paragraph{(2).} $y'' - y' - 2y = -2t + 4t^2$
\paragraph{Ans: } We know that our solution is the sum of the complementary and particular solution.
Solving the corresponding homogenous problem

\[
  y'' - y' -2y = 0 
\]

\paragraph{} Using $y = e^{rt}$.

\begin{align*}
    (r^2 - r - 2)e^{rt} &= 0\\
          (r^2 - r - 2) &= 0\\
          (r+1)(r-2) &= 0 \\
\end{align*}
\paragraph{} Therefore, we have
\[
    y = C_1 e^{2t} + C_2 e^{-t}  
\]

\paragraph{}We let our potential solution equal

\[
  y = A + Bt + Ct^2
\]
\paragraph{}such that our ODE equals
\begin{align*}
    (A+Bt+Ct^2)'' - (A+Bt+Ct^2)' - 2(A+Bt+Ct^2) &= -2t + 4t^2\\
    2C - B - 2Ct - 2A - 2Bt - 2Ct^2               &= -2t + 4t^2\\
    (-2A-B+2C) + (-2B-2C)t + (-2C)t^2 &= -2t + 4t^2
\end{align*}

\paragraph{}Giving us the system
\[
    \begin{bmatrix}
        -2 & -1 & 2 \\
        0 & -2 & -2 \\
        0 & 0 & -2
    \end{bmatrix} 
    \cdot 
    \phi
    =
    \begin{bmatrix}
    0\\
    -2\\
    4\\
    \end{bmatrix}
\]
\paragraph{}where
\[\phi = \begin{bmatrix}
    -3.5 \\
3 \\
-2
\end{bmatrix}\]
\paragraph{}Leaving us with our complete solution of 
\[
y= C_1e^{2t}+C_2e^{-t}-3.5+3t-2t^2 
\]

\paragraph{(3).}$y''+y'-6y=12e^{3t} +12e^{-2t}$
\paragraph{}As we did in (2). we solve the corresponding homogenous problem
\[
  y'' + y' - 6y = 0 
\]
\begin{align*}
    r^2 + r - 6 &= 0\\
    (r-3)(r+2) &= 0
\end{align*}
\paragraph{}Leaving us with 
\[
    y = C_1e^{3t}+C_2e^{-2t} 
\]
\paragraph{}We assume then our particular solution must take the form
\[
    y = Ae^{3t} + Be^{-2t}
\]
\paragraph{}Which means our ODE equals
\begin{align*}
    (Ae^{3t} + Be^{-2t})'' + (Ae^{3t}+Be^{-2t})' - 6(Ae^{3t}+Be^{-2t}) &= 12e^{3t}+12e^{-2t}\\
    9Ae^{3t}+4Be^{-2t} + 3Ae^{3t}-2Be^{-2t} - 6Ae^{3t}-6Be^{-2t} &= 12e^{3t} + 12e^{-2t}\\
    6Ae^{3t} -4Be^{-2t} &= 12e^{3t}+12e^{-2t}
\end{align*}
\paragraph{}Which is true if $A=2$, and $B = -3$.
\paragraph{}Leaving us with the complete solution
\[
    y = C_1e^{3t}+C_2e^{-2t}+2e^{3t}-3e^{-2t}
\]
\paragraph{(5)} $y''+2y'=3+4\sin(2t)$
\paragraph{}Solving corresponding homogenous problem

\begin{align*}
    r^2 + 2r &= 0\\
    r(r+2) &= 0
\end{align*}

\paragraph{}Leaving us with the complementary solution
\[
    y = C_1 + C_2e^{-2t} 
\]
\paragraph{}Because our nonhomogenous problem includes a sin function and 
a constant, we let our particular solution equal
\[
    y = A\cos(2t) + B\sin(2t) + Ct
\]
\paragraph{}Thus
\begin{align*}
    (A\cos(2t)+B\sin(2t)+Ct)'' + 2(A\cos(2t)+B\sin(2t)+Ct)' &= 3 + 4\sin(2t)\\
    -4A\cos(2t) -4B\sin(2t) - 4A\sin(2t) +4B\cos(2t) + 2C &= 3 + 4\sin(2t)\\
    (-4A + 4B)\cos(2t) + (-4A - 4B)\sin(2t) + 2C &= 3 + 4\sin(2t)
\end{align*}
\paragraph{}We conclude that $C = \frac{3}{2}$, and that the following system must hold
\[
  \begin{bmatrix}
      -4 & 4 \\
      -4 & -4 
  \end{bmatrix} 
  \cdot \phi
  =
  \begin{bmatrix}
    0\\
    4
  \end{bmatrix}
\]
\paragraph{}Which we solve to produce
\[
    \phi =
    \begin{bmatrix}
    -.5\\
    -.5
    \end{bmatrix}
\]
\paragraph{}Thus, our complete solution is
\[
    y = C_1 + C_2e^{-2t} - \frac{1}{2}\cos(2t)-\frac{1}{2}\sin(2t)+\frac{3}{2}t 
\]
\paragraph{(6).} $y''+2y'+y=2e^{-t}$
\paragraph{}Solving corresponding homogenous problem
\begin{align*}
    r^2 + 2r + 1 &= 0 \\
    (r+1)(r+1) &= 0
\end{align*}
\paragraph{}In the case of repeated roots, we let our solution equal
\[
    y = C_1e^{-t} + C_2te^{-t} 
\]
\paragraph{}Also in the case of repeated roots, we let our particular solution be
\[
    y = At^2e^{-t} 
\]
\paragraph{}Therefore
\begin{align*}
    (At^2e^{-t})'' +2(At^2e^{-t})' +(At^2e^{-t}) &= 2e^{-t}\\
    A(t^2e^{-t} -2te^{-t} + 2(-te^{-t}+e^{-t}))+2A(-t^2e^{-t} + 2te^{-t})+(At^2e^{-t}) &= 2e^{-t}\\
    At^2e^{-t}-2Ate^{-t}-2Ate^{-t}+2Ae^{-t}-2At^{2}e^{-t}+4Ate^{-t}+At^{2}e^{-t}&= 2e^{-t}\\
    2Ate^{-t} &= 2e^{-t}
\end{align*}
\paragraph{}Leaving us with $A=1$, and our complete solution being
\[
    y = C_1e^{-t}+C_2te^{-t}+t^2e^{-t} 
\]
\paragraph{(7).}$y''+y=3\sin(2t)+t\cos(2t)$
\paragraph{}Solving corresponding homogenous problem
\begin{align*}
    r^2 + 1 &= 0\\
\end{align*}
\paragraph{}Using Euler's formula
\[
    e^{i\phi} = \cos(\phi) + i\sin(\phi) 
\]
\paragraph{}We find the complementary solution
\[
    y = C_1\cos(t) + C_2\sin(t) 
\]
\paragraph{}We find that the cos function in our nonhomogenous problem has a linear coefficient. Thus, we
choose the following as our particular solution
\[
  y = A\sin(2t) + Bt\cos(2t) + C\sin(2t)
\]
\paragraph{}Making the left hand sum of our ODE equal to
\begin{align*}
    (A \sin 2t + Bt \cos 2t + C \sin 2t)'' + (A \sin 2t + Bt \cos 2t + C \sin 2t)\\
    = (2A \cos 2t + B(-2t \sin 2t + \cos 2t) + 2C \cos 2t)' \\+ (A \sin 2t + Bt \cos 2t + C \sin 2t)\\
    =(-4A \sin 2t + B(-4t \cos 2t -2 \sin 2t - 2\sin 2t) - 4C \sin 2t) \\+(A \sin 2t + Bt \cos 2t + C \sin 2t)\\
    =(-4A \sin 2t + -4Bt \cos 2t -4B \sin 2t - 4C \sin 2t) \\+(A \sin 2t + Bt \cos 2t + C \sin 2t)\\
    =(-3A -4B - 3C)\sin 2t -3Bt \cos 2t\\
\end{align*}
\paragraph{}Giving us
\[
    (-3A - 4B -3C)\sin2t - 3Bt\cos2t = 3\sin2t + t\cos2t
\]
\paragraph{}and the following system
\[
  \begin{bmatrix}
    -3 & -4 & -3\\
    0 & -3 & 0
  \end{bmatrix} 
  \cdot \phi
  = 
  \begin{bmatrix}
    3 \\ 1
  \end{bmatrix}
\]
\paragraph{}Using Octave, we find
\[
    \phi = 
  \begin{bmatrix}
       -0.2778\\
  -0.3333\\
    -0.2778\\ 
  \end{bmatrix} 
\]
\paragraph{}and a complete solution of
\[
    y = C_1\cos(t) + C_2\sin(t)-\frac{5}{9}\sin(2t)-\frac{1}{3}t\cos(2t) 
\]

\paragraph{(12).} $y''+4y=t^2+3e^t,\, y(0)=0, y'(0)=2$
\[
    r^2 + 4 = 0
\]
\[
   r = \pm2i 
\]
\paragraph{Complementary Solution:}
\[
  y = C_1 \cos2t + C_2 \sin2t 
\]
\paragraph{}We choose the following function as our potential particular solution:

\[
    y = (A + Bt + Ct^2) + De^{t} 
\]
\paragraph{}Which makes our ODE equal to

\begin{align*}
    ((A + Bt + Ct^2) + De^{t})'' + 4((A + Bt + Ct^2) + De^{t}) &= t^2 + 3e^t\\
    2C + De^t + 4A + 4Bt + 4Ct^2 &= t^2 + 3e^t\\
    (4A + 2C) + 4Bt + 4Ct^2 + 4De^t &= t^2 + 3e^t\\
\end{align*}
\paragraph{}And the following system

\[
  \begin{bmatrix}
    4 & 0 & 2 & 0\\
    0 & 4 & 0 & 0\\
    0 & 0 & 4 & 0\\
    0 & 0 & 0 & 5\\
  \end{bmatrix}
  \cdot \phi
  = 
  \begin{bmatrix}
    0\\0\\1\\3
  \end{bmatrix}
\]
\paragraph{}Using Octave
\[
  \phi =
  \begin{bmatrix}
    -0.1250\\
        0\\
   0.2500\\
   0.6000
  \end{bmatrix}
\]
\paragraph{}Therefore
\[
    y = C_1 \cos2t + C_2 \sin 2t -\frac{1}{8} + \frac{1}{4}t^2 + \frac{3}{5}e^t
\]
\paragraph{}and
\[
    y' = -2 C_1 \sin2t + 2 C_2 \cos 2t +\frac{1}{2}t + \frac{3}{5}e^t
\]
\paragraph{}using our initial conditions
\begin{align*}
    0 &= C_1  - \frac{1}{8} + \frac{3}{5}\\
    -\frac{19}{40} &= C_1\\
    2 &= 2C_2 + \frac{3}{5}\\
    \frac{7}{10} &= C_2
\end{align*}
\paragraph{}and the solution to our IVP is
\[
    y = -\frac{19}{40}\cos2t + \frac{7}{10}\sin 2t -\frac{1}{8} + \frac{1}{4}t^2 + \frac{3}{5}e^t
\]
\paragraph{(13).} $y''-2y'+y=te^t+4,\, y(0)=1, y'(0)=1$
\begin{align*}
    r^2 -2r + 1 &= 0\\
    (r-1)(r-1) &=0\\
\end{align*}
\paragraph{Complementary Solution:}
\[
    y = C_1e^{t} + C_2te^{t} 
\]
\paragraph{}Due to our repeated roots, we let our potential particular solution be
\[
    y = At^2e^2 + Bt^3e^t + C
\]
\paragraph{}And our ODE becomes equal to
\begin{align*}
    (At^2e^2 + Bt^3e^t + C)'' - 2( At^2e^2 + Bt^3e^t + C)' + ( At^2e^2 + Bt^3e^t + C)\\
    = te^t + 4\\
(\text{using special mathway and derivative calculator properties})\\
\vdots\\
    2Ae^t + 6Bte^t + C = te^t + 4
\end{align*}
\paragraph{}Giving us the following system
\[
  \begin{bmatrix}
    2 & 0 & 0\\
    0 & 6 & 0\\
    0 & 0 & 1\\
  \end{bmatrix}
  \cdot \phi
  \begin{bmatrix}
    0\\1\\4
  \end{bmatrix}
\]
\[
  \phi = 
  \begin{bmatrix}
    0\\0.1666\\4
  \end{bmatrix}
\]
\paragraph{}Therefore, our general solution is
\[
    y = C_1 e^{t} + C_2te^{t} + \frac{1}{6} t^3e^t + 4 
\]
\paragraph{}and
\[
    y' = C_1 e^{t} + C_2te^{t} + C_2e^{t} + \frac{1}{6}t^3e^t + \frac{1}{2}t^2e^t 
\]
\paragraph{}Applying initial conditions
\begin{align*}
    1 &= C_1+4\\
    -3 &= C_1
\end{align*}
\begin{align*}
    1 &= -3 + C_2\\
    4 &= C_2
\end{align*}
\paragraph{}The solution to our IVP is
\[
    y = -3e^{t} + 4te^{t} + \frac{1}{6} t^3e^t + 4  
\]
\paragraph{(14).}$y''+4y=3\sin(2t),\, y(0)=2, y'(0)=-1$
\begin{align*}
    r^2 + 4 = 0
\end{align*}
\[
  r = \pm 2i 
\]
\paragraph{Complementary Solution:}
\[
    y = C_1\cos2t + C_2\sin2t 
\]
\paragraph{}Let our potential particular solution be
\[
  y = A\sin2t + B\cos2t 
\]
\paragraph{}These are already part of the homogenous solution, so let's multiply by a factor of $t$.
\[
  y = At\sin2t + Bt\cos2t 
\]
\paragraph{}Which makes our ODE equal to
\begin{align*}
    (At\sin2t + Bt\cos2t)'' + 4(At\sin2t + Bt\cos2t) &= 3\sin2t\\
    (2At\cos2t + A\sin2t - 2Bt\sin2t + B\cos2t)' \\
    + 4At\sin2t + 4Bt\cos2t = 3\sin2t\\
    -4At\sin2t +2A\cos2t + 2A\cos2t -4Bt\cos2t -2B\sin2t-2B\sin2t \\
     + 4At\sin2t + 4Bt\cos2t = 3\sin2t\\
     4A\cos2t - 4B \sin2t = 3\sin2t
\end{align*}
\paragraph{}Giving us $A = 0$, and $B = -\frac{3}{4}$, and a general solution
\[
    y = C_1\cos2t + C_2 \sin2t -\frac{3}{4}t\cos2t
\]
\paragraph{}and
\[
    y' = -2C_1 \sin2t + 2C_2\cos2t + \frac{3}{2}\sin2t + \frac{3}{4}\cos2t
\]
\paragraph{}Applying initial conditions
\begin{align*}
    2 = C_1
\end{align*}
\begin{align*}
    -1 &= 2C_2 + \frac{3}{4}\\
    \frac{1}{8} &= C_2\\
\end{align*}

\paragraph{}Thus, the solution to our IVP is
\[
    y = 2\cos2t -\frac{1}{8}\sin2t -\frac{3}{4}t\cos2t 
\]
\paragraph{(4.)} $y''+y=\tan t, 0<t<\frac{\pi}{2}$
\paragraph{}Homogenous solution:
\[
  y = C_1 \cos t +C_2 \sin t 
\]
\paragraph{}Using variation of parameters
\begin{align*}
    ( C_1(t)\cos t +C_2(t)\sin t)'' + ( C_1(t)\cos t +C_2(t)\sin t) &= \tan t\\
                                                                    &\vdots \\
        C_1''(t)\cos t -2C_1'(t)\sin t +C_2''(t)\sin t + 2C_2'(t)\cos t &= \tan t
\end{align*}
\paragraph{I don't have time tonight to solve the rest, but I promise to go to the library tomorrow and finish this
:D}
\end{document}


