\documentclass{article}
\usepackage{amsmath, amsthm, amssymb, fancyhdr, tikz, enumitem, forest}
\usetikzlibrary{graphs}

\forestset{
  forest circles/.style={
    for tree={
      math content,
      draw,
      circle,
      minimum size=20pt,
      inner sep=0pt,
      text centered
    },
    no edge/.append style={draw=none}
  },
}

\pagestyle{fancy}
\lhead{CISC220 - HW4}
\begin{document}
\begin{enumerate}[label=(\arabic*).]
    \item (5 points) \textbf{Argue} that in a \emph{binary} tree of $N$ nodes, there are $N + 1$ null-pointer links representing
children. Note that each node has a left link and a right link pointing to the (possibly empty) left child
and the (possibly empty) right child, respectively.


\begin{center}
    \begin{forest}
      forest circles,
      [
        [
        [],
        [[, no edge] []]    
        ]
        [, no edge]
        []
      ]
    \end{forest}
\end{center}
    \begin{proof}\, 
        \paragraph{Theorem:} A tree of $N$ vertices has $N-1$ edges.
        \paragraph{}By the above theorem, and by properties of trees,
        we know that we have $N-1$ child nodes. For $N$ nodes, we
        have $2N$ links. We know that $N-1$ of these are \emph{not} null. 
        \[
            2N - (N-1) = N + 1
        \]
        \paragraph{}By this result, $N+1$ of these links must be null.
    \end{proof}
    \item (9 points) \textbf{Prove by induction} that the maximum number of nodes in a binary tree 
        of height $h$ is $2^{h+1} - 1$.
        \begin{proof}
            \textbf{Height of a Binary Tree:} The height of a binary tree is the number of edges
            between its root and its furthest leaf node.
            \paragraph{}Consider the tree which maximizes the number of nodes in a given binary tree 
            of height $h$ --- this tree must be a perfect binary tree.
            \paragraph{}Then, for a perfect binary tree of height $h$, the number of nodes for height 
            $h$ is equal to 
            
            \[
                \sum_{i=0}^{h} 2^i
            \]

            \paragraph{}as each level must have double the number of nodes as the previous level.
            \paragraph{}Let $h = 0$.
            \begin{align*}
                \sum_{i=0}^{0} 2^{i} &= 2^{0+1}-1\\
                1 &= 1
            \end{align*}
            \paragraph{}Then assume $h = k$.
            \begin{align*}
                \sum_{i=0}^{k} 2^i &= 2^{k+1}-1\\
                \sum_{i=0}^{k+1} 2^i &= 2^{(k+1)+1}-1\\
                \bigg(\sum_{i=0}^{k}2^i \bigg)+ 2^{k+1} &= 2^{k+2} -1\\
                2^{k+1}-1 + 2^{k+1} &= 2^{k+2} - 1\\
                2^{k+2}-1 &= 2^{k+2} -1 
            \end{align*}
        \end{proof}
    \item (10 points) (a). Show the result of inserting 3, 1, 4, 6, 9, 2, 5, 7 into an initially empty binary search
tree. (b). Show the result of deleting the root.

\vspace{5mm}

(a).\begin{center}
    \begin{forest}
      forest circles,
      [3,
        [1
            [,no edge], [2]
        ],
        [4
            [,no edge], [6
                [5], 
                [9
                    [7],
                    [, no edge]
                ]
            ]
        ]
      ]
    \end{forest}
\end{center}


(b).\begin{center}
    \begin{forest}
      forest circles,
      [4,
        [1
            [,no edge], [2]
        ],
        [6
            [5], 
            [9
                [7],
                [, no edge]
            ]
        ]
      ]
    \end{forest}
\end{center}
\item (6 points) Give the prefix, infix, and postfix expressions corresponding to the following tree.


\begin{center}
    \begin{forest}
      forest circles,
      [-
        [*
            [*
                [a],[b]
            ], 
            [+
                [c],[d]
            ]
        ],[e]
      ]
    \end{forest}
\end{center}
\paragraph{Prefix:}$$-**ab+cde$$ 
\paragraph{Infix:} $$((a\cdot b)\cdot(c + d)) - e$$
\paragraph{Postfix:}$$ab*cd+*e-$$

\end{enumerate}
\end{document}

