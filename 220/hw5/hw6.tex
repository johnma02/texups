\documentclass[letterpaper,12pt]{article}
\setlength{\headheight}{15pt}
\setlength{\marginparwidth}{0pt}
\setlength{\marginparsep}{0pt} % width of space between body text and margin notes
\setlength{\evensidemargin}{0.125in} % Adds 1/8 in. to binding side of all 
% even-numbered pages when the "twoside" printing option is selected
\setlength{\oddsidemargin}{0.125in} % Adds 1/8 in. to the left of all pages when "oneside" printing is selected, and to the left of all odd-numbered pages when "twoside" printing is selected
\setlength{\textwidth}{6.375in} % assuming US letter paper (8.5 in. x 11 in.) and side margins as above
\raggedbottom
\setlength{\parskip}{\medskipamount}
\usepackage{amsmath, amsthm, amssymb, fancyhdr, enumitem, tikz, pgfplots, forest}
\pgfplotsset{compat=1.18}
\pagestyle{fancy}
\lhead{CISC220 --- HW5}
\begin{document}
\begin{enumerate}[label=(\arabic*).]
    \item Derive a recurrence relation and give initial conditions for the
        number of bit strings of length $n$ without two consecutive 0s.
        \paragraph{Ans:}
        \paragraph{}Suppose you had a bit string of size $n = 1$. There are two possible strings:
        \[
            \texttt{0} \hspace{5mm} \texttt{1}
        \]
        \paragraph{}For a bit string of size $n = 2$, we have the following strings:
        \[
            01\hspace{5mm}
            11\hspace{5mm}
            10\hspace{5mm}
        \]
        \paragraph{}In general, we are concerned with the number of
        ways which we can choose the last character in the string, and the choice of 
        character. 
        \begin{itemize}
            \item If the last character was a 0, we may \emph{only} choose the next character
                to be a 1.
            \item It follows that if the last character was a 0, then the character before
                the last character \emph{must have been a 1}.
            \item Otherwise, we may choose either 1 or 0 as our next character.
        \end{itemize}
        \paragraph{}Then suppose that $T(n)$ represents the number of ways to form a bit string of length $n$.
        From our observations above, we know that $T(n)$ is composed of the number of ways to form
        a string of length $n-1$ ending in 1, and a string of length $n-1$ ending in 0.
        \paragraph{}However, we also know that a string ending in 0 is simply a string ending in 1
        with a 0 appended to its end.
        \paragraph{}In essence, we are looking for the number of ways to form a string of length
        $n-1$ ending in 1, and a string of length $n-2$ ending in 1.
        \[
            T(n) = T(n-1) + T(n-2) 
        \]
        \[
            T(1) = 2\,,T(2) = 3
        \]
        
    \item The number of bacteria in a colony doubles every hour. If the colony begins with 5 bacteria, denoted
as $B(0) = 5$, how many will be present in $n$ hours, $B(n)$?
\paragraph{Ans:}
\begin{align*}
    B(n) &= 2(B(n-1))\,, B(0) = 5\\
         &= 5\cdot 2^n
\end{align*}
    \item A rock band would like to tour $n$ cities. However, time will allow for visits to only $k$ cities. Because
time is short, the band members are not concerned about the order in which they visit the same $k$
cities. Let $g(n, k)$ be the number of groups of $k$ cities chosen from $n$. Derive $g(n, k)$ with three base
cases of $k = 0, k = n$, and $k > n$.
\paragraph{Ans:}
    \paragraph{}Consider a single city. The band may visit it, in this case, they will choose $k-1$ remaining
    cities from $n-1$, or they may not visit it, and they will choose $k$ cities from $n-1$.
    \[
        g(n,k) = g(n-1, k-1) + g(n-1, k)
    \]
    \paragraph{}Where if $k=0$, there is one way to choose 0 cities, if $k = n$, there is only one way to 
    choose all the cities, and if $k > n$, there are zero ways to choose more than $n$ cities.
    \item Write a recursive function to determine whether two trees are the same in terms of the data they store
and the order in which they store it.

    \begin{verbatim}
struct Tree {
    int key;
    Tree *left;
    Tree *right;
};
bool equal_trees(Tree *tree1, Tree *tree2){
    if(tree1 ^ tree2) return !(tree1 || tree2);
    if(tree1->key != tree2->key) return false;
    bool left = true; 
    if(tree1->left && tree2->left){
        left = equal_tree(tree1->left, tree2->left);
    }
    else{
        left = !(tree1->left || tree2->left);
    }
    bool right = true;
    if(tree1->right && tree2->right){
        right = equal_tree(tree1->right, tree2->right);
    }
    else{
        right = (tree1->right || tree2->right);
    }
    return (left && right);
} 
    \end{verbatim}
\item Show, number-by-number, the resulting AVL trees of inserting 2, 1, 4, 5, 9, 3, 6, 7 into an initially
empty AVL tree. Denote either single-rotation or double-rotation was used to rebalance the tree.

\forestset{
  forest circles/.style={
    for tree={
      math content,
      draw,
      circle,
      minimum size=20pt,
      inner sep=0pt,
      text centered
    },
    no edge/.append style={draw=none}
  },
}

\paragraph{2:}

\begin{center}
    \begin{forest}
      forest circles,
      [
        2 
      ]
    \end{forest}
\end{center}

\paragraph{1:}

\begin{center}
    \begin{forest}
      forest circles,
      [
        2,
        [1], [,no edge]
      ]
    \end{forest}
\end{center}

\paragraph{4:}

\begin{center}
    \begin{forest}
      forest circles,
      [
        2,
        [1], [4]
      ]
    \end{forest}
\end{center}

\paragraph{5:}

\begin{center}
    \begin{forest}
      forest circles,
      [
        2,
        [1], [4,
            [, no edge], [5]
            ]
      ]
    \end{forest}
\end{center}

\paragraph{9: Single Rotation}

\begin{center}
    \begin{forest}
      forest circles,
      [
        2,
        [1], [4,
            [, no edge], [5
                [, no edge],[9]
                ]
            ]
      ]
    \end{forest}
    \hspace{5mm}
    $\to$
    \hspace{5mm}
    \begin{forest}
      forest circles,
      [
        2,
        [1], [5,
            [4], [9]
            ]
      ]
    \end{forest}
\end{center}

\paragraph{3: Double Rotation}

\begin{center}
    \begin{forest}
      forest circles,
      [
        2,
        [1], [5,
            [4
                [3], [, no edge]
                ], 
            [9]
            ]
      ]
    \end{forest}
    \hspace{5mm}$\to$\hspace{5mm}
    \begin{forest}
      forest circles,
      [
        2,
        [1], [4,
            [3],
            [5
                [, no edge], [9]
                ]
            ]
      ]
    \end{forest}
    \hspace{5mm}$\to$\hspace{5mm}
    \begin{forest}
      forest circles,
      [
        4,
        [2
            [1],[3]
            ]
        [5
            [, no edge], [9]]
      ]
    \end{forest}
\end{center}
\paragraph{6: Double Rotation}
\begin{center}
    \begin{forest}
      forest circles,
      [
        4,
        [2
            [1],[3]
            ]
        [5
            [, no edge], [9
                [6], [, no edge]    
            ]
            ]
      ]
    \end{forest}
    \hspace{4mm}$\to$\hspace{4mm}
    \begin{forest}
      forest circles,
      [
        4,
        [2
            [1],[3]
            ]
        [5
        [,no edge][6
                [, no edge],[9]
                ]
            ]
      ]
    \end{forest}
    \hspace{4mm}$\to$\hspace{4mm}
    \begin{forest}
      forest circles,
      [
        4,
        [2
            [1],[3]
            ]
            [6
            [5],[9]
            ]
      ]
    \end{forest}
\end{center}
\paragraph{7:}
\begin{center}
 \begin{forest}
      forest circles,
      [
        4,
        [2
            [1],[3]
            ]
        [6
        [5]
            [9
    [7], [, no edge]
            ]
            ]
      ]
    \end{forest}
\end{center}

\item The following program, generating the output \texttt{‘11 W World’}, shows the use of operators \texttt{size()},
    \texttt{[]}, and \texttt{substr()} on the string class of the C++ standard library.
    \begin{verbatim}
#include <iostream>
using namespace std;
int main(){
    string str="Hello World";
    cout << str.size()<<’ ’<< str[6]<<’ ’<<str.substr(6)<<’\n’;
    return 0;
}
    \end{verbatim}
Complete the following recursive C++ function \texttt{reverse()} by using the
operators \texttt{size()}, \texttt{[],} and \texttt{substr()}, so that given a 
string of characters, \texttt{reverse()} prints characters of the string in reverse
order. For instance, given \texttt{‘hello’, reverse()} prints \texttt{‘olleh.’}
\begin{verbatim}
#include <iostream>
#include <string>
using namespace std;
void reverse(string str){
    cout << str[str.size()-1];
    if(str.size()<1)return;
    reverse(str.substr(0, str.size()-1)); 
}
int main(){
    string a = "hello world";
    reverse(a);
    return 0;
}
\end{verbatim}
\end{enumerate}
\end{document}



